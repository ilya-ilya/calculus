\documentclass[a4paper, 12pt, num=28]{listok}
\usepackage{bm}
%\usepackage{scalerel}
%\usepackage{tikz}
%\usepackage{epigraph}
%\usepackage{multicol}
%\usetikzlibrary{matrix}
%\usetikzlibrary{calc}
%\usetikzlibrary{arrows}

\DeclareMathOperator{\id}{id}

%\newcommand*{\hm}[1]{#1\nobreak\discretionary{}{\hbox{$\mathsurround=0pt #1$}}{}} %перед знаком для его дублирования при переносе формулы

\begin{document}
\title{Топология прямой: Начало}
\maketitle{}
\begin{problem}[ (лемма о вложенных отрезках)]\label{subsec}
	Имеется последовательность отрезков, каждый из которых содержится в предыдущем.
	Может ли пересечение всех этих отрезков быть пустым?
\end{problem}
\begin{problem}
	Изменится ли ответ, если отрезки заменить интервалами?
\end{problem}
\begin{problem}
	Группа естествоиспытателей в течении 6 часов наблюдала за (неравномерно) ползущей улиткой так,
	что она всё это время была под присмотром.
	Каждый наблюдатель следил за улиткой ровно 1 час без перерывов и зафиксировал, что она проползла за этот
	час ровно 1 метр. Могла ли улитка за время всего эксперимента проползти
	\begin{probparts}
		\item 5м?
		\item 10м?
		\item 12м?
	\end{probparts}
\end{problem}
\begin{problem}[ (лемма о конечном покрытии)]\label{coversec}
	Докажите, что в любом покрытии отрезка интервалами найдётся конечный набор интервалов, покрывающий весь отрезок\footnote{%
		Подсказка: воспользуйтесь леммой о вложенных отрезках.
	}.
\end{problem}
\begin{problem}
	Из любого ли покрытия отрезка интервалами можно удалить часть так, чтобы оставшиеся интервалы тоже составляли покрытие,
	но каждую точку накрывало бы не более двух из них?
\end{problem}
\begin{problem}
	Изменится ли что-нибудь в предыдущих двух задачах, если отрезок заменить интервалом?
\end{problem}
\begin{problem}
	Всегда ли из покрытия отрезка
	\begin{probparts}
		\item конечным;
		\item любым множеством
	\end{probparts}
	содержащихся внутри него отрезков можно выкинуть часть отрезков так,
	чтобы оставшиеся по-прежнему покрывали исходный отрезок и их суммарная длина не превышала бы его удвоенной длины?
\end{problem}
\begin{definition}
	Для любого положительного числа $\epsilon$ будем называть \textit{$\epsilon$-окрестностью} точки $x_0 \in \R$ интервал
	$
		D_{\epsilon}(x_0) \coloneq \left ( x_0 - \epsilon, x_0 + \epsilon \right )
		= \left \{ x \in \R \mid |x - x_0| < \epsilon \right \}
	$
	длины $2 \epsilon$ с центром в $x_0$.
\end{definition}
\begin{definition}
	Точка $a \in \R$ называется \textit{внутренней точкой} множества $M \subset \R$,
	если у неё есть $\epsilon$-окрестность, целиком содержащаяся в $M$.
\end{definition}
\begin{definition}
	Множество называется \textit{открытым}, если все его точки --- внутренние.
	Иначе говоря, $U \subset \R$ открыто, если
	$\forall{u \in U} \exists{\epsilon > 0} D_{\epsilon}(u) \subset U$.
	Пустое множество тоже, по определению, считается открытым.
\end{definition}
\begin{problem}
	Убедитесь, что промежутки
	\begin{probparts}
		\item $(-\infty, a)$;
		\item $(a, +\infty)$;
		\item $(a, b)$;
	\end{probparts}
	открыты.
\end{problem}
\begin{problem}
	Можно ли разбить интервал в объединение двух непересекающихся открытых множеств?
\end{problem}
\begin{problem}
	Докажите, что объединение любого набора открытых множеств открыто.
\end{problem}
\begin{problem}
	\begin{probparts}
		\item Докажите, что пересечение конечного набора открытых множеств открыто.
		\item Так ли это для пересечений бесконечных наборов?
	\end{probparts}
\end{problem}
\begin{problem}
	Докажите, что всякое открытое множество на прямой представляет собой объединение конечного или счётного набора попарно непересекающихся интервалов,
	в числе которых допускаются и неограниченные интервалы типа $(-\infty, a)$, $(a, +\infty)$ и $(-\infty, +\infty)$.
\end{problem}
\begin{definition}
	Точка $a \in M \subset \R$ называется \textit{изолированной точкой} множества $M$,
	если некоторая её $\epsilon$-окрестность не содержит никаких других точек из $M$.
\end{definition}
\begin{definition}
	Множество, все точки которого изолированы, называется \textit{дискретным}.
\end{definition}
\begin{problem}
	Может ли бесконечное дискретное множество быть
	\begin{probparts}
		\item ограниченным?
		\item несчётным?
	\end{probparts}
\end{problem}
\end{document}
