\documentclass[a4paper, 12pt, num=31]{listok}
\usepackage{bm}
%\usepackage{scalerel}
%\usepackage{tikz}
\usepackage{epigraph}
%\usepackage{multicol}
%\usetikzlibrary{matrix}
%\usetikzlibrary{calc}
%\usetikzlibrary{arrows}

\DeclareMathOperator{\id}{id}

%\newcommand*{\hm}[1]{#1\nobreak\discretionary{}{\hbox{$\mathsurround=0pt #1$}}{}} %перед знаком для его дублирования при переносе формулы

\begin{document}
\title{Счёт}
\maketitle{}
\begin{problem}
    Найдите предел последовательности:
    \begin{probparts}
        \item $1 + {0{,}1}^n$;
        \item $\frac{5n + 4}{6n + 15}$;
        \item $\frac{2^n - 3^n}{2^n + 3^n}$;
        \item $1 + q + q^2 + \cdots + q^n$;
        \item $\sqrt[n]{2}$.
    \end{probparts}
\end{problem}
\begin{problem}
    Вычислите пределы (при $n \to + \infty$) или докажите расходимость последовательностей:
    \begin{probparts}
        \item $\frac{{(2n - 3)}^{20} {(3n + 2)}^{30}}{{(2n + 1)}^{50}}$,
        \item $\frac{\sqrt[3]{n^2}\sin{n!}}{2n - 3}$,
        \item $\frac{\sqrt{1 + 2n} - 3}{\sqrt n - 2}$,
        \item $\sqrt{n + 1} - \sqrt{n - 1}$,
        \item $\sqrt{n^2 + 3n} - \sqrt{n^2 - 5n}$,
        \item $\sqrt2\sqrt[4]2\sqrt[8]2\dots\sqrt[2^n]2$,
        \item $\underbrace{\sqrt{2\sqrt{2\sqrt{\dots\sqrt 2}}}}_{\text{$n$ радикалов}}$,
        \item $\underbrace{\sqrt{2 + \sqrt{2 + \sqrt{2 + \cdots + \sqrt 2}}}}_{\text{$n$ радикалов}}$,
    \end{probparts}
\end{problem}
\begin{problem}[(у)]
    Докажите, что любая ограниченная последовательность содержит сходящуюся подпоследовательность.
\end{problem}
\begin{problem}
\begin{probparts}
    \item Верно ли, что последовательность $a_n = \cfrac{\cos2}2 + \cfrac{\cos 2^2}{2^2} + \cdots + \cfrac{\cos 2^n}{2^n}$ монотонна при $n > 1$?
    \item Докажите, что данная последовательность имеет предел.
\end{probparts}
\end{problem}
\begin{problem}
    Последовательность $\{a_n\}$ задана рекуррентно: $a_1 = 4$, $a_{n + 1} = \frac{a_n + 3}5$.
    \begin{probparts}
        \item Докажите, что данная последовательность имеет предел и
        \item найдите этот предел.
    \end{probparts}
\end{problem}
\begin{problem}[(Итерационная формула Герона)]
    Пусть $x_1$ --- произвольное положительное число, $x_{n + 1} = \frac12 \left ( x_n + \frac a{x_n} \right )$, где $a > 0$.
    Докажите, что тогда при $n \ge 2$ последовательность $\{x_n\}$
    \begin{probparts}
        \item убывающая,
        \item ограниченная снизу числом $\sqrt a$,
        \item имеет предел,
        \item причём $\lim\limits_{n \to + \infty} x_n = \sqrt a$.
        \item Для $a = 10$, $x_1 = 1$ найдите такое $n$, что $|x_n - \sqrt a| < 10^{-5}$.
    \end{probparts}
\end{problem}
\begin{problem}[(Теорема Штольца)]
    Пусть последовательность $\{a_n\}$ является возрастающей бесконечно большой последовательностью с положительными членами,
    а последовательность $\{b_n\}$ такова, что $\lim\limits_{n\to+\infty}\frac{b_{n + 1} - b_n}{a_{n + 1} - a_n} = L$.
    Докажите, что тогда существует предел $\lim\limits_{n\to+\infty}\frac{b_n}{a_n} = L$.
\end{problem}
\begin{problem}
    Пусть $P(n) = a_k n^k + \cdots + a_1 n + a_0$ и $Q(n) = b_l n^l + \cdots + b_1 n + b_0$  два многочлена степеней $k$ и $l$ соответственно,
    причём $\forall{n \in \N} Q(n) \ne 0$.
    Докажите, что\footnote{Последнее надо понимать в том смысле, что $\frac{P (n)}{Q (n)}$ --- бесконечно большая.}
    \[
        \lim_{n \to + \infty} \frac{P(n)}{Q(n)} = \left \{ 
        \begin{aligned}
            0, & \quad\text{если $l > k$,} \\
            \frac{a_k}{b_l}, & \quad\text{если $l = k$,}\\
            +\infty, & \quad\text{если $l < k$.}
        \end{aligned}
        \right . 
    \]
\end{problem}
\begin{problem}
    Докажите, что $\lim\limits_{n \to + \infty} \cfrac{1^k + 2^k + \cdots + n^k}{n^{k + 1}} = \cfrac1{k + 1}$.
\end{problem}
\begin{problem}[(Уравнение Кеплера)]\footnote{%
        Данное уравнение появилось в работах И. Кеплера (1571--1630) при изучении движения планет по эллиптическим орбитам (т.н. задача двух тел).}
    Для решения уравнения $x - \alpha \sin x = C$, где $0 < \alpha < 1$,
    строят последовательность $\{x_n\}$ следующим образом: $x_1 = C$, $x_{n + 1} = C + \alpha \sin x_n$.
    Докажите, что
    \begin{probparts}
        \item уравнение Кеплера имеет не более одного корня;
        \item последовательность $\{x_n\}$ имеет предел;
        \item $\lim\limits_{n\to +\infty} x_n$ является решением уравнения Кеплера.
    \end{probparts}
\end{problem}
\begin{problem}[(\textup{[:|||:]})]
    Докажите равенства:
    \begin{probparts}
        \item $\lim\limits_{n \to + \infty} \cfrac{n^k}{a^n} = 0$, $a > 1$, $k \in \N$;
        \item $\lim\limits_{n \to + \infty} \cfrac{a^n}{n!} = 0$;
        \item $\lim\limits_{n \to + \infty} {n^k}{q^n} = 0$, $q < 1$, $k \in \N$;
        \item $\lim\limits_{n \to + \infty} \sqrt[n]a = 1$, $a > 0$;
        \item $\lim\limits_{n \to + \infty} \sqrt[n]n = 1$;
        \item $\lim\limits_{n \to + \infty} \cfrac1{\sqrt[n]{n!}} = 0$;
    \end{probparts}
\end{problem}
\theme{Число Непера}
\begin{definition}
    Введём несколько обозначений:
    \begin{align*}
        a_n &= {\left ( 1 + \frac 1n \right )}^n &
        b_n &= {\left ( 1 + \frac 1n \right )}^{n + 1} &
        c_n &= 1 + \frac1{1!} + \frac1{2!} + \frac1{3!} + \cdots + \frac1{n!}
    \end{align*}
\end{definition}
\begin{problem}
\begin{probparts}
    \item Докажите, что при фиксированном $i > 0$ последовательность $\frac{n - i}n$ убывает.
    \item Докажите, что $a_n$ возрастает\footnote{<<В чём сила, брат?>>}.
    \item Докажите, что $b_n$ убывает\footnote{Почему летают самолёты?}.
    \item Докажите, что $a_n$ и $b_n$ ограничены.
    \item Докажите, что $\lim\limits_{n \to +\infty} a_n = \lim\limits_{n \to + \infty} b_n$.
\end{probparts}
\end{problem}
\begin{definition}
    Пределом последовательностей $a_n$ и $b_n$ является число $e = 2{,}7182818284590\dots$, которое называется \textit{числом Непера}.
\end{definition}
\begin{problem}
    Докажите, что последовательность $c_n$
    \begin{probparts}
        \item имеет предел и 
        \item этот предел равен $e$.
    \end{probparts}
\end{problem}
\begin{problem}[\hard]
\begin{probparts}
    \item Докажите, что для последовательности $c_n$ верно неравенство $0 < e - c_n < \frac1{n\cdot n!}$.
    \item С помощью данного неравенства докажите, что число $e$ является иррациональным.
\end{probparts}
\end{problem}
\begin{problem}
    Найдите пределы
    \begin{probparts}
        \item $\lim\limits_{n \to + \infty} {\left ( 1 + \frac 1n \right )}^{3n}$;
        \item $\lim\limits_{n \to + \infty} {\left ( 1 + \frac 3n \right )}^{n}$;
        \item $\lim\limits_{n \to + \infty} {\left ( 1 - \frac 1n \right )}^{n}$.
    \end{probparts}
\end{problem}
\begin{problem}
    Докажите неравенства:
    \begin{probparts}
        \item ${\left ( \frac n e \right )}^n < n! < e {\left ( \frac n2 \right )}^n$;
        \item $\forall{\alpha \in \R} e^\alpha > 1 + \alpha$.
    \end{probparts}
\end{problem}
\theme{Простые предельные теоремы}
\begin{problem}[(Теорема Бернулли)]
    Пусть $A_{m, n, p}$ --- событие, что среди $n$ испытаний Бернулли с вероятностью успеха $p$ ровно $m$ закончились успешно.
    Докажите, что для любого $\epsilon > 0$ существует предел
    \[
        \lim_{n \to + \infty} \Prob{A_{m, n, p} \mid |\frac mn - p| < \epsilon} = 1.
    \]
\end{problem}
\begin{problem}[(Закон Больших Чисел)]
    Пусть $X_1, \dots, X_n, \dots$ --- последовательность попарно независимых случайных величин,
    для всех $i$ $\Expect{X_i^2}$ --- конечно, $m = \Expect{X_i}$, $\sigma^2 = \Expect{X_i^2} - {(\Expect{X_i})}^2$.
    Рассмотрим \textit{выборочное среднее}:
    \[
        {\closure{X}}^{(n)} = \frac 1n \sum_{i = 1}^n X_i.
    \]
    Докажите, что для любого $\epsilon > 0$ существует предел
    \[
        \lim_{n \to +\infty} \Prob{|{\closure{X}}^{(n)} - m| \ge 0} = 0.
    \]
\end{problem}
\end{document}
