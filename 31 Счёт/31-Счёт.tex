\documentclass[a4paper, 12pt, num=31]{listok}
\usepackage{bm}
%\usepackage{scalerel}
%\usepackage{tikz}
\usepackage{epigraph}
%\usepackage{multicol}
%\usetikzlibrary{matrix}
%\usetikzlibrary{calc}
%\usetikzlibrary{arrows}

\DeclareMathOperator{\id}{id}

%\newcommand*{\hm}[1]{#1\nobreak\discretionary{}{\hbox{$\mathsurround=0pt #1$}}{}} %перед знаком для его дублирования при переносе формулы

\begin{document}
\title{Счёт}
\maketitle{}
\begin{problem}
    Найдите предел последовательности:
    \begin{multienum}{3}
        \item $1 + {0{,}1}^n$;
        \item $\frac{5n + 4}{6n + 15}$;
        \item $\frac{2^n - 3^n}{2^n + 3^n}$;
        \item $1 + q + q^2 + \cdots + q^n$;
        \item $\sqrt[n]{2}$.
    \end{multienum}
\end{problem}
\begin{problem}
    Огромная счётная задача.
\end{problem}
\begin{problem}[(Итерационная формула Герона)]
    ЗАДАКА!
\end{problem}
\begin{problem}
    Штольц и $C^O$.
\end{problem}
\begin{problem}
    Кеплер
\end{problem}
\begin{problem}
    Баяны (найти \LaTeX{} символ для баяна.)
\end{problem}
\theme{Число Непера}
Тутошки надо расскрыть скобки и посмотреть на всё кроме констант.
\end{document}
