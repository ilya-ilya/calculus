\documentclass[a4paper, 12pt, num=25]{listok}
\usepackage{bm}
%\usepackage{scalerel}
%\usepackage{tikz}
%\usepackage{epigraph}
%\usepackage{multicol}
%\usetikzlibrary{matrix}
%\usetikzlibrary{calc}
%\usetikzlibrary{arrows}

\DeclareMathOperator{\id}{id}

%\newcommand*{\hm}[1]{#1\nobreak\discretionary{}{\hbox{$\mathsurround=0pt #1$}}{}} %перед знаком для его дублирования при переносе формулы

\begin{document}
\title{Бесконечно Большие Последовательности}
\maketitle{}
\begin{definition}
	Последовательность называется \textit{неограниченной}, если она не является ограниченной.
\end{definition}
\begin{definition}
	Последовательность $\{a_n\}$ называется \textit{бесконечно большой}, если для любого числа $C > 0$ найдется такое число $k$,
	что при всех натуральных $n$, больших $k$, будет верно неравенство $|a_n| > C$.
\end{definition}
\begin{problem}
	Сформулируйте, не используя отрицания, определения неограниченной и бесконечно большой последовательностей с помощью кванторов.
	Верно ли, что любая бесконечно большая последовательность является неограниченной? А наоборот?
	Рассмотрим множество неограниченных и множество бесконечно больших последовательностей. Пересекаются ли они?
	Является ли одно из них подмножеством другого?
\end{problem}
\begin{problem}
	Сформулируйте, не используя отрицания, определение последовательности, которая
	\begin{probparts}
		\item не является неограниченной;
		\item не является бесконечно большой.
	\end{probparts}
	Запишите их с помощью кванторов.
\end{problem}
\begin{problem}
	Последовательность $\{a_n\}$ бесконечно большая. Верно ли, что она монотонная?
	А последовательность $\{|a_n|\}$?
\end{problem}
\begin{problem}
	Является ли бесконечно большой последовательность, равная
	\begin{probparts}
		\item сумме;
		\item разности;
		\item произведению;
		\item отношению бесконечно больших последовательностей?
	\end{probparts}
\end{problem}
\begin{problem}
	Изменятся ли ответы на вопросы предыдущей задачи, если везде заменить бесконечно большие последовательности на неограниченные?
\end{problem}
\begin{definition}
	Пусть $\{n_i\}$  возрастающая последовательность натуральных чисел.
	Последовательность $\{b_i\}$, где $b_i = a_{n_i}$, называется \textit{подпоследовательностью} последовательности $\{a_n\}$.
\end{definition}
\begin{problem}
\begin{probparts}
	\item Докажите, что любая подпоследовательность ограниченной последовательности ограничена.
\end{probparts}
	Останется ли верным аналогичное утверждение в случае
	\begin{probparts}[resume]
		\item монотонной;
		\item неограниченной;
		\item бесконечно большой последовательности?
	\end{probparts}
\end{problem}
\begin{problem}
	Докажите, что
	\begin{probparts}
		\item любая последовательность содержит монотонную подпоследовательность;
		\item любая неограниченная последовательность содержит бесконечно большую подпоследовательность.
	\end{probparts}
\end{problem}
\begin{problem}
\begin{probparts}
	\item Докажите, что для любой ограниченной последовательности существует отрезок длины 1,
	в котором находится бесконечно много членов этой последовательности.
	\item Верно ли, что если для некоторой последовательности такого отрезка длины 1 найти нельзя, то эта последовательность  бесконечно большая?
\end{probparts}
\end{problem}
\begin{problem}
	Докажите, что для любой ограниченной монотонной последовательности найдется отрезок длины 1,
	в котором находятся все члены этой последовательности, начиная с некоторого номера.
\end{problem}
\begin{problem}
	Придумайте две различные последовательности, являющиеся подпоследовательностями друг друга.
\end{problem}
\begin{problem}
\begin{probparts}
	\item Последовательность $\{a_n\}$ такова, что $|a_{n+1} - a_n| \le \frac1{2^n}$ при любом натуральном $n$.
	Может ли эта последовательность быть неограниченной?
	\item Тот же вопрос, если $|a_{n+1} - a_n| \le \frac1n$.
\end{probparts}
\end{problem}
\begin{problem}
\begin{probparts}
	\item Придумайте такую последовательность натуральных чисел, что любая последовательность натуральных чисел является ее подпоследовательностью.
	\item Можно ли решить аналогичную задачу, если натуральные числа всюду заменить на рациональные числа?
	\item А на действительные?
\end{probparts}
\end{problem}
\begin{problem}
	Существует ли такая последовательность целых чисел, что любое натуральное число представимо в виде разности двух членов этой последовательности,
	причем единственным образом?
\end{problem}
\begin{problem}
	Докажите, что найдется такое число $a > 0$,
	при котором дробные части всех чисел последовательности $\{a_n\}$ принадлежат отрезку $[1/3; 2/3]$.
\end{problem}
\begin{problem}
\begin{probparts}
	\item Докажите, что у всякой последовательности длины $n^2 + 1$ существует монотонная подпоследовательность длины $n + 1$.
	\item Останется ли верным утверждение задачи в случае последовательности длины $n^2$ при больших значениях $n$?
\end{probparts}
\end{problem}
\end{document}
