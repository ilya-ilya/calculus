\documentclass[a4paper, 12pt, num=21]{listok}
\usepackage{bm}
%\usepackage{scalerel}
%\usepackage{tikz}
%\usepackage{epigraph}
%\usepackage{multicol}
%\usetikzlibrary{matrix}
%\usetikzlibrary{calc}
%\usetikzlibrary{arrows}

\DeclareMathOperator{\id}{id}

%\newcommand*{\hm}[1]{#1\nobreak\discretionary{}{\hbox{$\mathsurround=0pt #1$}}{}} %перед знаком для его дублирования при переносе формулы

\begin{document}
\title{Комплексная арифметика}
\maketitle{}
\begin{definition}
	\textit{Компл\'ексными\footnote{В отличие от обеда, который к\'омплексный.} числами} будем называть формальные записи $a + b i$, где $a, b \in \R$.
	Множество таких чисел будем обозначать $\Cx$.
	Символ $i$ называется \textit{мнимой единицей}, число $a$ "--- \textit{действительной частью} комплексного числа $z$ и обозначается $\Re z$,
	число $b$ "--- \textit{мнимой частью} комплексного числа $z$ и обозначается $\Im z$.
	Комплексные числа складываются, вычитаются и перемножаются по тем же законам, по которым производятся операции с
	многочленами, при этом полагается, что $i^2 = -1$.	
\end{definition}
\begin{definition}
	\textit{Сопряжённым} к комплексному числу $z$ называют комплексное число $\bar{z} = \Re z - i \Im z$.
\end{definition}
\textit{Везде далее под числом будем понимать комплексное число, если не оговорено иное.}
\begin{problem}
Даны числа $z_1 = 1 + i$, $z_2 = 4 − 3i$.
Найдите
\begin{enumerate}
\begin{multicols}{5}
	\item $\Re z_2$;
	\item $\Im z_2$;
	\item $z_1 + z_2$;
	\item $z_1 - z_2$;
	\item $z_1z_2$;
	\item $z_1/z_2$;
	\item $1/z_1$;
	\item $z_1^3$;
	\item $z_1^{2017}$.
\end{multicols}
\end{enumerate}
\end{problem}
6  2. Найдите два комплексных числа, сумма которых равна 4, а произведение 5.
6  3. а) Найдите какое-нибудь число z, т.ч. Im z 6= 0, Im z 3 = 0.уравнения z 3 = 1.
б) Найдите все корни
Определение 6.2. Сопряженным к комплексному числу z называют комплексное число
z = Re z − i Im z.
6  4◦ . Докажите, что
а) (z) = z;
 в) Im(zz) = 0;
 д) zw = z w;
б) Im(z + z) = 0;
 г) z ± w = z ± w;
 е) z/w = z/w.
6  5◦
 . Для числа z = a + bi напишите формулы, по которым можно найти противоположное
−z и обратное z −1 ему числа.
6  6◦ . Найдите общую формулу для частного (a + ib)/(c + id). (Для получения формулы
удобно воспользоваться сопряженными числами.)
6  7. Докажите, что два числа с отличной от нуля мнимой частью являются сопряженными
тогда и только тогда, когда их сумма и произведение являются действительными числами.

\end{document}
