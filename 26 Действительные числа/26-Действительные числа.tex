\documentclass[a4paper, 12pt, num=26]{listok}
\usepackage{bm}
%\usepackage{scalerel}
%\usepackage{tikz}
%\usepackage{epigraph}
%\usepackage{multicol}
%\usetikzlibrary{matrix}
%\usetikzlibrary{calc}
%\usetikzlibrary{arrows}

\DeclareMathOperator{\id}{id}

%\newcommand*{\hm}[1]{#1\nobreak\discretionary{}{\hbox{$\mathsurround=0pt #1$}}{}} %перед знаком для его дублирования при переносе формулы

\begin{document}
\title{Действительные числа}
\maketitle{}
\begin{definition}
	\textit{Бесконечная десятичная дробь (БДД)} --- это конечная влево и бесконечная вправо последовательность десятичных цифр вида
	$\pm\overline{b_{-n}\ldots b_{-1}b_0{,}b_1b_2\ldots b_m\ldots}$.
	Куски слева и справа от запятой называются \textit{целой} и \textit{дробной} частями данной БДД.
	Если БДД имеет лишь конечное множество ненулевых цифр, то она называется \textit{конечной};
	у такой БДД все цифры правее некоего разряда --- нули, их обычно не пишут.
\end{definition}
\end{document}
