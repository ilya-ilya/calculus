\documentclass[a4paper, 12pt, num=26]{listok}
\usepackage{bm}
%\usepackage{scalerel}
%\usepackage{tikz}
%\usepackage{epigraph}
%\usepackage{multicol}
%\usetikzlibrary{matrix}
%\usetikzlibrary{calc}
%\usetikzlibrary{arrows}

\DeclareMathOperator{\id}{id}

%\newcommand*{\hm}[1]{#1\nobreak\discretionary{}{\hbox{$\mathsurround=0pt #1$}}{}} %перед знаком для его дублирования при переносе формулы

\begin{document}
\title{Действительные числа}
\maketitle{}
\begin{definition}
	\textit{Бесконечная десятичная дробь (БДД)} --- это конечная влево и бесконечная вправо последовательность десятичных цифр вида
	$\pm\overline{b_{-n}\ldots b_{-1}b_0{,}b_1b_2\ldots b_m\ldots}$.
	Куски слева и справа от запятой называются \textit{целой} и \textit{дробной} частями данной БДД.
	Если БДД имеет лишь конечное множество ненулевых цифр, то она называется \textit{конечной};
	у такой БДД все цифры правее некоего разряда --- нули, их обычно не пишут.
\end{definition}
\begin{definition}
	\textit{Множество действительных чисел} $\R$ получается из множества всех БДД путём отождествления друг с другом некоторых пар
	БДД по следующему правилу: две БДД, по определению, изображают одно и то же действительное число, если и только если они одного знака
	и существует номер $i \in \Z$, такой что цифры этих дробей во всех разрядах левее $i$-того совпадают, в $i$-том разряде различаются ровно на единицу,
	и дробь с большей $i$-той цифрой имеет справа от $i$-того разряда одни нули, а с меньшей — одни девятки.

	Иными словами, $\overline{b_\beta \ldots b_{i - 1}b_{i}9999\ldots} = \overline{b_\beta \ldots b_{i - 1}(b_i + 1)0000\ldots}$ как действительные числа.
	Если действительное число имеет два представления в виде БДД, то представление, оканчивающееся нулями,
	мы будем называть \textit{стандартным} и использовать по умолчанию именно его.
\end{definition}
\begin{problem}
	Докажите, что множество $\R$ несчётно.
\end{problem}
\begin{definition}
	Положительные БДД \textit{больше} отрицательных.
	Для сравнения двух положительных БДД их стандартные представления записывают друг под другом,
	выровнив по запятой; \textit{меньшей} считается БДД с меньшей самой левой из несовпадающих цифр.
	Для отрицательных БДД наоборот: $-a < -b \Leftrightarrow a > b$.
\end{definition}
\begin{definition}
	Число $b \in \R$ называется \textit{верхней гранью} для данного множества действительных чисел $M \subset \R$,
	если $\forall{x \in M} b \ge x$. Множество $M \subset \R$ называется \textit{ограниченным сверху}, если у него есть верхняя грань.
	Аналогично определяются \textit{нижняя грань} и \textit{ограниченность снизу}.
	Множества ограниченные одновременно и сверху и снизу называются просто \textit{ограниченными}.
\end{definition}
\begin{problem}
	Не употребляя отрицаний, дайте определения
	\begin{probparts}
		\item числа, не являющегося нижней гранью данного множества $M \subset \R$;
		\item множества, неограниченного сверху;
		\item неограниченного множества.
	\end{probparts}
	Запишите их с помощью кванторов.
\end{problem}
\begin{definition}
	Число $m \in M$, такое что $\forall{x \in M} m \ge x$, называется \textit{максимальным элементом} множества $М \subset \R$.
	Максимальный элемент в множестве нижних граней данного множества $M$ называется \textit{точной нижней гранью} (сокращённо: тнг) и
	обозначается $\inf M$ (от латинского <<infimum>>).
\end{definition}
\begin{problem}
	Дайте определение минимального элемента и точной верхней грани (сокращённо: твг) данного множества $M \subset \R$
	(она обозначается $\sup M$ --- от латинского <<supremum>>).
\end{problem}
\begin{problem}
	Каждое ли ограниченное сверху подмножество в $\R$ имеет максимальный элемент?
\end{problem}
\begin{problem}
	Вычислите твг множеств:
	\begin{probenum}
		\item $\{0{,}3; 0{,}33; 0{,}333; 0{,}3333 ; 0{,}33333 ; \ldots \}$;
		\item сумм $1 + q + q^2 + \cdots + q^m$ с фиксированным $0< q< 1$ и любым $m \in \N$.
	\end{probenum}
\end{problem}
\begin{problem}
	Верно ли, что любая БДД $b \in \R$ есть твг множества:
	\begin{probenum}
		\item всех БДД, меньших $b$;
		\item всех конечных БДД, меньших $b$;
		\item всех конечных БДД, являющихся начальными кусками $b$.
	\end{probenum}
\end{problem}
\begin{problem}
	\begin{probparts}
		\item Докажите, что $b = \sup M$, ограниченной снизу $M \subset \R$, обладает свойством \textit{предельности}:
		для лбого $\epsilon > 0$ найдётся $x \in M$ такое, что $b - \epsilon \le x \le b$.
		\item Докажите, что это свойство является достаточным.
		То есть, что любая верхняя грань обладающая этим свойством будет точной.
	\end{probparts}
\end{problem}
\begin{problem}[ (теорема о полноте)]\label{completness}
	Докажите, что каждое ограниченное сверху подмножество в $\R$ имеет точную верхнюю грань,
	а каждое ограниченное снизу --- точную нижнюю.
\end{problem}
\begin{problem}[ (сложение и умножение)]
	Суммой БДД $a, b \in \R$ называется твг чисел вида $\alpha + \beta$,
	где $\alpha < a$ и $\beta < b$ --- всевозможные \textit{конечные} БДД.
	Аналогично, произведение $a\cdot b$ двух \textit{положительных} БДД --- это твг чисел вида
	$\alpha \cdot \beta$ с \textit{конечными положительными} $\alpha < a$ и $\beta < b$
	(на неположительные БДД произведение распространяется по стандартному правилу <<$-$>> на <<$-$>> даёт <<$+$>>).
	Докажите, что
	\begin{probparts}
		\item эти определения корректны (т. е. нужные твг существуют),
		\item дают для конечных $a$ и $b$ то же, что и раньше.
		\item Проверьте, что для любых $a, b, c \in \R$ верно, что $a(b + c) = ab + ac$.
	\end{probparts}
\end{problem}
\begin{problem}[ (корни)]\label{roots}
	Пусть $b \in \R$ есть твг конечных БДД $\beta$ с $\beta^2 < 5$.
	Докажите, что $b$ существует и вычислите $b^2$.
\end{problem}
\begin{definition}
	БДД с нулевой дробной частью называются \textit{целыми}.
	На координатной прямой множество целых БДД $\Z \subset \R$ получается откладыванием от $0$ всевозможных целых кратных единичного отрезка.
	БДД $r \in \R$ называется рациональной, если $\exists{k,m \in \Z} kr = m$ в $\R$.
	Подмножество рациональных БДД $\Q \subset \R$ на числовой прямой изображается откладыванием от $0$ всевозможных отрезков,
	соизмеримых\footnote{два отрезка называются соизмеримыми, если они допускают общую единицу измерения --- третий отрезок,
	который целое число раз укладывается в каждом из них} с единичным.
\end{definition}
\begin{problem}
	Докажите, что конечные БДД рациональны.
\end{problem}
\begin{problem}
	Верно ли, что $\forall{\alpha \in \R, \epsilon > 0} \exists{r_1, r_2 \in \Q} \alpha - \epsilon < r_1 < \alpha < r_2 < \alpha + \epsilon$?
\end{problem}
\begin{definition}
	Разбиение $\Q = A_1 \sqcup A_2$ в объединение двух непересекающихся подмножеств,
	таких что $a_1 < a_2$ для любых $a_1 \in A_1, a_2 \in A_2$, называется \textit{сечением} множества $\Q$.
\end{definition}
\begin{problem}
	Докажите, что для любого сечения $\Q = A_1 \sqcup A_2$ имеет место равенство $\sup A_1 = \inf A_2$.
\end{problem}
\begin{problem}
	Докажите, что для любого сечения $\Q = A_1 \sqcup A_2$ выполняется ровно одна из трёх возможностей:
	либо в $A_1$ есть максимальный элемент, либо в $A_2$ есть минимальный элемент,
	либо действительное число $\sup A_1 = \inf A_2$ иррационально.
\end{problem}

\textit{%
	Забудем на время про БДД, будем понимать $\Q$ как множество обыкновенных дробей
	$p/q, p \in \Z, q \in \N$, и назовём действительным числом Дедекинда любое сечение
	$\Q = A_1 \sqcup A_2$, в котором $A_i$ не имеет максимального элемента.
}

\begin{problem}
	Дайте определение
	\begin{probparts}
		\item суммы и
		\item произведения
	\end{probparts}
	действительных чисел Дедекинда\footnote{это должно быть сечение нужного типа}.
\end{problem}
\begin{problem}
	Установите сохраняющую арифметические операции биекцию (\textit{изоморфизм}) между множеством
	дедекиндовых действительных чисел и множеством действительных чисел, определённоv посредством БДД\footnote{%
	Подсказка: сопоставьте каждому классу эквивалентных БДД $\alpha \in \R$ сечение
	$\Q = \{ q \in \Q \mid q < \alpha \} \cup \{ q \in \Q \mid q \ge \alpha \}$.}
\end{problem}
\begin{problem}
	Исходя только из дедекиндова определения действительных чисел\footnote{т.е. не пользуясь БДД, но в терминах сечений}
	\begin{probparts}
		\item докажите теорему о полноте (ср. с задачей~\ref{completness}) и
		\item определите квадратные корни (ср. с задачей~\ref{roots}).
	\end{probparts}
\end{problem}
\end{document}
