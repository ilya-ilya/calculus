% \iffalse meta-comment
%
% Copyright (C) 2018 by Ilya
% -----------------------------------
%
% This file may be distributed and/or modified under the
% conditions of the LaTeX Project Public License, either version 1.3
% of this license or (at your option) any later version.
% The latest version of this license is in:
%
%    http://www.latex-project.org/lppl.txt
%
% and version 1.3 or later is part of all distributions of LaTeX
% version 2005/12/01 or later.
%
% \fi
%
% \iffalse
%<*driver>
\ProvidesFile{listok.dtx}
%</driver>
%<common>\NeedsTeXFormat{LaTeX2e}[2005/12/01]
%<listok>\ProvidesClass{listok}
%<ustn>\ProvidesClass{ustn}
%<test>\ProvidesClass{test}
%<written>\ProvidesClass{written}
%<a4>\ProvidesFile{a4.clo}
%<a5>\ProvidesFile{a5.clo}
%<lua>\ProvidesFile{lua.clo}
%<pdf>\ProvidesFile{pdf.clo}
%<*common>
	[2019/01/09 v2.1  Наши крутые шаблоны для листков]
%</common>
%
%<*driver>
\documentclass[a4paper, 11pt]{ltxdoc}
\usepackage[TU]{fontenc}
\usepackage[lutf8]{luainputenc}
\usepackage{fontspec}
\setmainfont{Times New Roman}
\usepackage[russian]{babel}
\usepackage[a4paper, height = 270mm, left = 3cm, right = 1cm, centering]{geometry}
\usepackage{longtable}
\usepackage{tabu}
\usepackage{multirow}
\usepackage{amsmath, amssymb, amsfonts, amsthm}
\usepackage{microtype}
\EnableCrossrefs
\CodelineIndex
\RecordChanges
\begin{document}
\DocInput{listok.dtx}
\end{document}
%</driver>
% \fi
%
% \CheckSum{0}
%
%
% \GetFileInfo{listok.dtx}
%
% \DoNotIndex{\begin, \else, \end, \endinput, \fi, \forloop, \fill}
% \DoNotIndex{\;, \\, \centering, \endquotation, \quotation}
% \DoNotIndex{\AtBeginShipout, \AtBeginShipoutBox, \box, \copy}
% \DoNotIndex{\circ, \geqslant}
%
% \title{Семейство классов \textsf{listok} \thanks{Этот документ относится к \textsf{listok}~\fileversion, от \filedate}}
% \author{Илья Райко \\ \texttt{rayko\_i@179.ru}}
%
% \maketitle
%
% \begin{abstract}
% 	Классы, используемый нашей командой для вёрстки листков.
% \end{abstract}
%
% \section{Введение}
%
% На данный момент у нас есть классы \textsf{listok}, \textsf{written}, \textsf{ustn}, \textsf{test}.
% \textsf{listok}  используется для вёрстки листков,
% \textsf{written} и \textsf{ustn} используются для вёрстки письменных и устных собеседований,
% \textsf{test} используется для контрольных работ.
%
% Разница между ними описана в следующей табличке
%
% \begin{longtabu} to \textwidth {|c|c|c|c|c|}
% \hline
%  & \textsf{listok} & \textsf{test} & \textsf{written} & \textsf{ustn} \\
% \hline
% \multirow{2}{*}{Заголовок} & \multicolumn{4}{c|}{\texttt{\textbackslash{}title\{Title\}}} \\
% \cline{2-5}
% & {\Large \textit{ \textbf{ \underline{Title}}}} &  {\Large \textit{ \textbf{ \underline{Title. Вариант №|num| }}}} & \multicolumn{2}{c|}{\Large \textit{ \textbf{ \underline{\dots{}  собеседование}}}}\\
% \hline
% \multirow{2}{*}{Задача} & \multicolumn{4}{c|}{\texttt{\textbackslash{}begin\{problem\}[Text]}} \\
% \cline{2-5}
% & \textbf{Задача |num|.|problem|}\textit{Text} & \multicolumn{3}{c|}{\textbf{Задача |problem|}\textit{Text}} \\
% \hline
% \multirow{2}{*}{Пример} & \multicolumn{4}{c|}{\texttt{\textbackslash{}begin\{example\}[Text]}} \\
% \cline{2-5}
% & \textbf{Пример |num|.|example|}\textit{Text} & \multicolumn{3}{c|}{\textbf{Пример |example|}\textit{Text}} \\
% \hline
% \multirow{2}{*}{Теорема} & \multicolumn{4}{c|}{\texttt{\textbackslash{}begin\{theorem\}[Text]}} \\
% \cline{2-5}
% & \textbf{Теорема |num|.|theorem|} \textit{(Text)} & \multicolumn{3}{c|}{\textbf{Теорема |theorem|} \textit{(Text)}} \\
% \hline
% \end{longtabu}
%
% При этом у классов есть ещё некоторые индивидуальные особенности, которые будут описаны ниже.
%
% \section{Параметры классов}
% \subsection{Формат бумаги}
%    \begin{macrocode}
%<common>\RequirePackage{kvoptions}
%<common>\newcommand{\@paperfile}{}
%<common>\DeclareOption{a4paper}{\renewcommand{\@paperfile}{a4.clo}}
%    \end{macrocode}
% С |a4paper| всё просто --- напечатает одну копию на листе A4, ориентированном вертикально.
% Ширина текста будет 179mm, а высота 267mm.
% Текст выровнен по центру (вертикально и горизонтально).
%    \begin{macrocode}
%<*a4>
	\RequirePackage[a4paper, width = 179mm, height = 267mm, footskip = 0mm, centering, includeheadfoot]{geometry}
	\PassOptionsToClass{a4paper,oneside}{article}
%</a4>
%<common>\DeclareOption{a5paper}{\renewcommand{\@paperfile}{a5.clo}}
%    \end{macrocode}
% Чуть сложнее с |a5paper|.
% Он нужен для того, чтобы напечать два вертикальных A5 на одном горизонтальном A4.
% Признаться честно, я уже забыл, что делают конкретные команды, но это, кажется, работает.
%    \begin{macrocode}
%<*a5>
	\setlength\paperheight {210mm}
	\setlength\paperwidth  {148mm}
	\PassOptionsToClass{a5paper,oneside}{article}
	\RequirePackage{pgfpages}
	\RequirePackage{atbegshi}
	\setlength\hoffset{.4in}
	\setlength\oddsidemargin{-1in}
	\setlength\textwidth{130mm}
	\setlength\voffset{-25mm}
	\setlength\textheight{185mm}
	\pgfpagesuselayout{2 on 1}[a4paper,landscape,border shrink=5mm]
	\AtBeginShipout{%
		\pgfpagesshipoutlogicalpage{1}\copy\AtBeginShipoutBox%
		\pgfpagesshipoutlogicalpage{2}\box\AtBeginShipoutBox%
		\pgfshipoutphysicalpage%
	}
%</a5>
%    \end{macrocode}
% \subsection{Дата и номер листка}
% |date| задаёт дату листочка в углу.
% Сейчас значение по умолчанию --- сегодня.
% Бейте меня палками, если до сих пор не взлетает.
%    \begin{macrocode}
%<*common>
\DeclareStringOption[\DTMToday]{date}
%    \end{macrocode}
% |num| --- номер листочка. Влияет на номер задач.
%    \begin{macrocode}
\DeclareStringOption[1]{num}
%    \end{macrocode}
% Все остальные опции передаются классу |article|.
%    \begin{macrocode}
\DeclareOption*{\PassOptionsToClass{\CurrentOption}{article}}
\ProcessOptions
\ProcessKeyvalOptions*
\LoadClass{article}
\input{\@paperfile}
\RequirePackage{iftex}
%    \end{macrocode}
% \section{Движки}
% Вот тут начинается цирк с движками.
% Проблема заключается в том, что традиционный Web2C движок использует кодировки отличные от utf8 (OT2, T2A, T2B, T2C, etc.)
% Чем это плохо?
% В OT2 был очень странный порядок сортировки кириллических букв (Э, Ю, Ж, Й, Ё, Я, А, Б, Ц, Д, Е, Ф и т.д.),
% что выливается в отвратительные списки и предметные указатели.
% 
% Даже игнорируя это факт, мы получаем две проблемы:
% старый Web2C движок генерирует странный слой текст в pdf (зюковки и кракозябры вместо русского текста),
% можно использовать только шрифты поддерживающие кодировки T2.
% Проблемы в том, что был только один такой шрифт, в котором русские буквы имеют равную высоту.
% Этот шрифт распространялся пакетом |pscyr|, который не обновляется с конца 2000-х годов.
% И тут нас постигает совсем проблема: |pscyr| сегодня поддерживается только дистрибутивом MiK\TeX.
%
% Поэтому предлагается использовать расширенный движок Lua\TeX.
% Если не вдаваться в подробности, то имеет место простая формула:
% \begin{multline*}
% 	\text{Lua\TeX} = \text{pdf\TeX} + \text{Поддержка Unicode и OpenType шрифтов} +\\
% 	+ \text{e-\TeX} + \text{Omega} + \text{встроенный интерпретатор lua}
% \end{multline*}
% 
% Корректная работа классов семейства |listok| гарантируется при использовании Lua\LaTeX.
% В теории можно использовать и другие движки, но на свой страх и риск (на момент 19.01.2019 я не мог скомпилировать русский текст с pdf\LaTeX).
%
% Движок определяется автоматически по имени запускаемой программы.
%    \begin{macrocode}
\ifLuaTeX
	\input{lua.clo}
%</common>
%<*lua>
	\RequirePackage[TU]{fontenc}
	\RequirePackage[lutf8]{luainputenc}
	\RequirePackage{fontspec}
	\setmainfont{CMU Serif}
%</lua>
%<common>\else
%<pdf>\RequirePackage[utf8]{inputenc}
%<pdf>\RequirePackage{cmap}
%<*common>
	\input{pdf.clo}
\fi
%    \end{macrocode}
% \section{Фишки}
% Далее идут подключённые пакеты
%    \begin{macrocode}
\RequirePackage[russian]{babel}
\RequirePackage{enumitem}
\RequirePackage{textcomp, multicol}
\RequirePackage{mathtext}
\RequirePackage{indentfirst}
\RequirePackage{amsmath, amssymb, amsfonts, amsthm}
\RequirePackage{mathtools, mathabx}
\RequirePackage{epstopdf}
\RequirePackage{graphicx}
\RequirePackage{forloop}
\RequirePackage{datetime2}
\RequirePackage{microtype}
\relax
\tolerance 4000
%    \end{macrocode}
%|\listok@name| задаёт имя листка, затем используется командой |\maketitle| для заголовка.
%    \begin{macrocode}
%<listok>\newcommand{\listok@name}{Листок №\listok@num}
%<test>\newcommand{\listok@name}{Вариант №\listok@num}
%<written>\newcommand{\listok@name}{}
%<ustn>\newcommand{\listok@name}{Фамилия, Имя}
\renewcommand{\maketitle}{%
	\begin{center}%
	{\Large \textit{\textbf{\underline{%
%<listok>\@title % Лорен ипсум
%<test>\@title. \listok@name % Контрольная по Лорен ипсум. Вариант №N 
%<written>Письменное собеседование
%<ustn>Устное собеседование
	}}}}
	\end{center}
}
%    \end{macrocode}
%    \begin{macrocode}
\DTMsetup{datesep=.}
\DTMsetstyle{ddmmyyyy}
\renewcommand{\thefootnote}{\fnsymbol{footnote}}
\renewcommand{\@oddhead}{
	\vbox{\hbox to\textwidth{\listok@name\hfil \strut
	\listok@date
	}\hrule}
}
\renewcommand{\@oddfoot}{}
%    \end{macrocode}
%
% \subsection{Трансокеанизация}
% \begin{longtabu} to \textwidth {|c|c|}
% \hline
% Команда & Вывод \\
% \hline
% \texttt{\textbackslash{}emptyset} & $\varnothing$ \\
% \hline
% \texttt{\textbackslash{}le} & $\leqslant$ \\
% \hline
% \texttt{\textbackslash{}ge} & $\geqslant$ \\
% \hline
% \texttt{\textbackslash{}epsilon} & $\varepsilon$ \\
% \hline
% \texttt{\textbackslash{}phi} & $\varphi$ \\
% \hline
% \texttt{\textbackslash{}N} & $\mathbb{N}$ \\
% \hline
% \texttt{\textbackslash{}Z} & $\mathbb{Z}$ \\
% \hline
% \texttt{\textbackslash{}Q} & $\mathbb{Q}$ \\
% \hline
% \texttt{\textbackslash{}R} & $\mathbb{R}$ \\
% \hline
% \texttt{\textbackslash{}Cx} & $\mathbb{C}$ \\
% \hline
% \texttt{\textbackslash{}Re z} & $\operatorname{Re} z$ \\
% \hline
% \texttt{\textbackslash{}Im z} & $\operatorname{Im} z$ \\
% \hline
% \texttt{\textbackslash{}Zm\{p\}} & $\mathbb{Z}_{p}$ \\
% \hline
% \texttt{\textbackslash{}Prob\{Event\}} & $\mathbb{P} \left [ Event \right ] $ \\
% \hline
% \texttt{\textbackslash{}ProbV\{\xi\}\{Event\}} & $\mathbb{P}_{\xi} \left [ Event \right ] $ \\
% \hline
% \texttt{\textbackslash{}Expect\{\xi\}} & $\mathbb{E} \xi $ \\
% \hline
% \texttt{\textbackslash{}pt\{Text\}} & $\text{Text}^\circ$ \\
% \hline
% \texttt{Text\textbackslash{}point} & Text${}^\circ$ \\
% \hline
% \texttt{Text\textbackslash{}hard} & Text${}^\ast$ \\
% \hline
% \end{longtabu}
%    \begin{macrocode}
\renewcommand{\emptyset}{\varnothing}
\renewcommand{\le}{\leqslant}
\renewcommand{\ge}{\geqslant}
\renewcommand{\epsilon}{\varepsilon}
\renewcommand{\phi}{\varphi}
\newcommand{\N}{\mathbb{N}}
\newcommand{\Z}{\mathbb{Z}}
\newcommand{\Q}{\mathbb{Q}}
\newcommand{\R}{\mathbb{R}}
\newcommand{\Cx}{\mathbb{C}}
\renewcommand{\Re}[1]{\operatorname{Re}{#1}}
\renewcommand{\Im}[1]{\operatorname{Im}{#1}}
\newcommand{\Zm}[1]{\mathbb{Z}_{#1}}
\renewcommand{\Prob}[1]{\mathbb{P}\left [ #1 \right ]}
\newcommand{\ProbV}[2]{\mathbb{P}_{#1}\left [ #2 \right ]}
\DeclareMathOperator{\Expect}{\mathbb{E}}
\newcommand{\wdt}{\widetilde}
\newcommand{\pt}[1]{$\text{#1}^\circ$}
\newcommand{\point}{${}^\circ$}
\newcommand{\hard}{${}^\ast$}
%    \end{macrocode}
% \subsection{Переопределение кванторов}
% Проблема: команды \verb|\exists| и \verb|\forall| задают просто символ, а не оператор или что-то ещё.
% Из этого следует, что на печати надо самим заботиться о пробелах после кванторной связки.
% Поэтому команды переопределены и ставят \verb|\;| после связки.
%
% К тому же доопределён макрос \verb|\existsone|.
%
% Команды \verb|\existssym| и \verb|\forallsym| задают символы кванторов в старом смысле.
%    \begin{macrocode}
\let\existssym\exists
\renewcommand*{\exists}[1]{\existssym #1~\;~}
\newcommand*{\existsone}[1]{\existssym! #1~\;~}
\let\forallsym\forall
\renewcommand*{\forall}[1]{\forallsym #1~\;~}
%    \end{macrocode}
% \subsection{Остальное}
% \DescribeEnv{abstract}
% Abstract имеет смысл использовать под разные комментарии.
% Текст печатается по центру курсивом.
%    \begin{macrocode}
\renewenvironment{abstract}{\quotation\itshape\centering}{\endquotation}
%    \end{macrocode}
% \DescribeEnv{theorem}\DescribeEnv{theorem*}
% \DescribeEnv{lemma}\DescribeEnv{lemma*}
% \DescribeEnv{proposition}\DescribeEnv{proposition*}
% \DescribeEnv{corollary}\DescribeEnv{corollary*}
% \DescribeEnv{definition}\DescribeEnv{definition*}
% \DescribeEnv{note}\DescribeEnv{note*}
% Для вёрстки теорем, лемм, утверждений, следствий, определений и замечаний определены окружения.
% Вариант со звёздочкой делает нумерованную теорему, etc.
%    \begin{macrocode}
\newtheorem{theorem}{Теорема}
\newtheorem*{theorem*}{Теорема}
\newtheorem{lemma}{Лемма}
\newtheorem*{lemma*}{Лемма}
\newtheorem{proposition}{Утверждение}
\newtheorem*{proposition*}{Утверждение}
\newtheorem{corollary}{Следствие}
\newtheorem*{corollary*}{Следствие}
\theoremstyle{definition}
\newtheorem*{definition}{Определение}
\newtheorem{note}{Замечание}
\newtheorem*{note*}{Замечание}
%    \end{macrocode}
% \DescribeEnv{problem}\DescribeEnv{problem*}
% \DescribeEnv{example}\DescribeEnv{example*}
% Новый стиль для \textsf{amsthm} теорем приследует следующую цель.
% Хотим отрисовывать звёздочку около сложной задачи и кружочек около обязательной.
% Для этого будем использовать аргумент, который обычно используют для названия теорем!
%
% Можно использовать и для примеров.
%    \begin{macrocode}
\newtheoremstyle{problem}{0pt}{0pt}{\normalfont}{}{\bfseries}{.}{ }{\thmname{#1}\thmnumber{ #2}\textit{#3}}
\theoremstyle{problem}
\newtheorem{example}{Пример}
\newtheorem*{example*}{Пример}
\newtheorem{problem}{Задача}
\newtheorem*{problem*}{Задача}
\renewcommand \theproblem {%
%<listok>	\listok@num.\@arabic\c@problem
%<common&!listok>\@arabic\c@problem
}
\renewcommand \theexample {%
%<listok>	\listok@num.\@arabic\c@example
%<common&!listok>\@arabic\c@example
}
\renewcommand \thetheorem {%
%<listok>	\listok@num.\@arabic\c@theorem
%<common&!listok>\@arabic\c@theorem
}
%    \end{macrocode}
% \DescribeMacro{\problems}
% |\problems| есть переписка с |\section*| из класса \textsf{article}.
% \DescribeMacro{\theme}
% |\theme| --- аналог |\subsection*|  из класса \textsf{article}.
% Рекомендуется использовать, если задачи явно сгруппированы по какой-то теме.
% Команды \verb|\problems| и \verb|theme| есть только в классе |listok|
%    \begin{macrocode}
%<*listok>
\newcommand \problems{\@startsection{section}{1}{\z@}{-3.5ex \@plus -1ex \@minus -.2ex}{2.3ex \@plus.2ex}{\normalfont\Large\bfseries\centering}*{Задачи}}
\newcommand \theme{\@startsection{subsection}{2}{\z@}{-3.25ex\@plus -1ex \@minus -.2ex}{1.5ex \@plus .2ex}{\normalfont\large\bfseries}*}

%</listok>
%    \end{macrocode}
% \DescribeEnv{vartab}
% Окружение |vartab|\marg{amount} необходимо для набора таблиц различного размера.
% amount задаёт количество столбцов.
%    \begin{macrocode}
\newenvironment{vartab}[1]
{
    \begin{tabular}{*{#1}{|c} |c| } \hline
}{
    \end{tabular}
}
%    \end{macrocode}
% \DescribeMacro{\conduit}
% |\conduit| печатает кондуит.
% В случае устного собеседования --- это табличка в пять строк и |\c@problem| столбцов (то есть по одному столбцу на задачу).
% (Три оценки за каждую задачу и подпись).
% На листочке и контрольной --- табличка с критериями оценки.
%    \begin{macrocode}
%<*ustn>
\newcounter{colidx}
\newcommand \conduit {

\vspace*{\fill}
\begin{center}
\begin{Large}
\begin{vartab}{\c@problem}
\hline
\forloop{colidx}{1}{\not{\value{colidx} > \c@problem}}{\; \arabic{colidx} \; &} \; $\Sigma$ \; \\ \hline
\forloop{colidx}{1}{\not{\value{colidx} > \c@problem}}{&} \\ \hline
\forloop{colidx}{1}{\not{\value{colidx} > \c@problem}}{&} \\ \hline
\forloop{colidx}{1}{\not{\value{colidx} > \c@problem}}{&} \\ \hline
\forloop{colidx}{1}{\not{\value{colidx} > \c@problem}}{&} \\
\forloop{colidx}{1}{\not{\value{colidx} > \c@problem}}{&} \\
\forloop{colidx}{1}{\not{\value{colidx} > \c@problem}}{&} \\ \hline
\end{vartab}
\end{Large}
\end{center}
}
%</ustn>
%<*listok|test>
\newcommand \conduit [4]{
\begin{center}
\begin{tabular}{|c|c|c|c|}
\hline
\multicolumn{4}{|c|}{Критерии оценок} \\
\hline
<<5>> & <<4>> & <<3>> & <<2>> \\ \hline
#1 задач & #2 задач & #3 задач & #4 задач \\ \hline
\end{tabular}
\end{center}
}
%</listok|test>
%    \end{macrocode}
% Нумерация пунктов по умолчанию кириллическими полужирными буквами с точкой (ie \textbf{а.}, \textbf{б.}, \textbf{в.})
%    \begin{macrocode}
%<*common>
\AddEnumerateCounter{\Asbuk}{\@Asbuk}{Ы}
\AddEnumerateCounter{\asbuk}{\@asbuk}{ы}
\setlist[itemize]{nolistsep}
\setlist[enumerate]{nolistsep, label = \textbf{\asbuk*.}}
\newenvironment{multienum}[1]
{
	\begin{enumerate}
	\begin{multicols}{#1}
}{
	\end{multicols}
	\end{enumerate}
}
\endinput
%</common>
%    \end{macrocode}
% \PrintChanges
% \PrintIndex
% \Finale
\endinput
