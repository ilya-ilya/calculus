\documentclass[a4paper, 12pt, num=29]{listok}
\usepackage{bm}
%\usepackage{scalerel}
%\usepackage{tikz}
%\usepackage{epigraph}
%\usepackage{multicol}
%\usetikzlibrary{matrix}
%\usetikzlibrary{calc}
%\usetikzlibrary{arrows}

\DeclareMathOperator{\id}{id}

%\newcommand*{\hm}[1]{#1\nobreak\discretionary{}{\hbox{$\mathsurround=0pt #1$}}{}} %перед знаком для его дублирования при переносе формулы

\begin{document}
\title{Топология прямой: Возрождение легенды}
\maketitle{}
\begin{definition}
	Точка $x_0 \in \R$ называется \textit{предельной} точкой множества $M \subset \R$,
	если любая её $\epsilon$-окрестность $D_{\epsilon}(x_0)$ содержит какую-нибудь точку $x \in M$, отличную от $x_0$.
\end{definition}
\begin{problem}
	Постройте бесконечное множество $M \subset \R$, множество предельных точек которого:
	\begin{probparts}
		\item пусто;
		\item состоит из одной точки;
		\item состоит из двух точек;
		\item совпадает с $\Z \subset \R$.
	\end{probparts}
\end{problem}
\begin{problem}
	Может ли множество предельных точек быть множеством всех чисел вида $1/k$ с $k\in\N$?
\end{problem}
\begin{problem}
	Может ли бесконечное ограниченное множество не иметь предельных точек?
\end{problem}
\begin{definition}
	Множество $Z \subset \R$ называется \textit{замкнутым}, если оно содержит все свои предельные точки (в частности, если их у него вообще нет).
\end{definition}
\begin{definition}
	Непустые ограниченные замкнутые множества называются \textit{компактами}.
\end{definition}
\begin{problem}
	Докажите, что отрезок и прямая --- замкнуты, а интервал и луч --- нет.
\end{problem}
\begin{problem}
\begin{probparts}
	\item Бывают ли дискретные незамкнутые множества?
	\item А бесконечные дискретные компакты?
\end{probparts}
\end{problem}
\begin{problem}
\begin{probparts}
	\item Разбивается ли отрезок в объединение двух непересекающихся непустых замкнутых множеств?
	\item Перечислите все одновременно открытые и замкнутые подмножества в $\R$.
\end{probparts}
\end{problem}
\begin{problem}
	Докажите, что множество $Z \subset \R$ замкнуто тогда и только тогда, когда его дополнение $\R \setminus Z$ открыто.
\end{problem}
\begin{problem}
	Докажите, что пересечение любого набора замкнутых множеств замкнуто.
\end{problem}
\begin{problem}
\begin{probparts}
	\item Докажите, что объединение конечного набора замкнутых множеств замкнуто.
	\item Так ли это для бесконечных наборов?
\end{probparts}
\end{problem}
\begin{problem}[ (лемма о вложенных компактах)]
	Докажите, что любая последовательность вложенных компактов $K_1 \supset K_2 \supset K_3 \supset \ldots$
	имеет непустое пересечение: $\bigcap K_n \ne \emptyset$.
\end{problem}
\begin{problem}[ (лемма о конечном покрытии)]
	Докажите, что непустое $K \subset \R$ компакт тогда и только тогда, когда любое его покрытие интервалами содержит конечное подпокрытие\footnote{%
		Подсказка: Сравните эти две задачи с задачами~\ref{subsec} и~\ref{coversec}.
	}.
\end{problem}
\theme{Канторово множество}
\begin{definition}
	Определим по индукции последовательность компактов $K_n \subset \left [ 0, 1 \right ]$ следующим образом.
	Каждый $K_n$ является объединением $2^n$ отрезков: $K_{n + 1}$ получается из $K_n$ выкидыванием открытой
	(т. е. без концевых точек) средней трети из каждого отрезка, входящего в $K_n$, а $K_0 = \left [ 0, 1 \right ]$.
	Пересечение $\mathfrak{K} \coloneq \bigcap_n K_n$ называется \textit{канторовым множеством}\footnote{%
	в честь Георга Кантора
	}.
	Иначе $\mathfrak{K} \subset \left [ 0, 1 \right ]$ можно определить как множество всех чисел $x \in \left [ 0, 1 \right ]$,
	которые в троичной системе счисления можно записать при помощи БДД, не содержащей 1, причём не требуется, чтобы эта запись была стандартной
	(т. е. она может оканчиваться на одни двойки).
\end{definition}
\begin{problem}
	Докажите, что канторово множество имеет мощность континуума, но покрывается конечным набором интервалов со сколь угодно малой суммой длин.
\end{problem}
\begin{problem}
	Содержит ли канторово множество:
	\begin{probparts}
		\item интервалы?
		\item изолированные точки?
	\end{probparts}
\end{problem}
\theme{Конденсация несчётных множеств}
\begin{definition}
	Элемент $a \in M$ произвольного несчётного множества $M \subset \R$ называется его точкой \textit{конденсации}, если
	при всех $\epsilon > 0$ $D_{\epsilon}(a) \cap M$ несчётно.
\end{definition}
\begin{problem}[\hard]
	Докажите, что множество точек $b \in M$, не являющихся точками конденсации, конечно или счётно.
\end{problem}
\begin{problem}[\hard]
	Может ли множество точек конденсации иметь изолированные точки?
\end{problem}
\begin{problem}[\hard]
	Докажите, что множество точек конденсации замкнуто.
\end{problem}
\begin{problem}[\hard]
	Докажите, что любое замкнутое подмножество в $\R$ или пусто, или конечно, или счётно, или имеет мощность континуума.
\end{problem}
\end{document}
