\documentclass[a4paper, 12pt, num=21, date=02.09.2019]{listok}
\usepackage{bm}
%\usepackage{scalerel}
%\usepackage{tikz}
%\usepackage{epigraph}
%\usepackage{multicol}
%\usetikzlibrary{matrix}
%\usetikzlibrary{calc}
%\usetikzlibrary{arrows}

\DeclareMathOperator{\id}{id}

%\newcommand*{\hm}[1]{#1\nobreak\discretionary{}{\hbox{$\mathsurround=0pt #1$}}{}} %перед знаком для его дублирования при переносе формулы

\begin{document}
\title{Комплексная арифметика}
\maketitle{}
\begin{definition}
	\textit{Компл\'ексными\footnote{В отличие от обеда, который к\'омплексный.} числами} будем называть формальные записи $z = a + b i$, где $a, b \in \R$.
	Множество таких чисел будем обозначать $\Cx$.
	Символ $i$ называется \textit{мнимой единицей}, число $a$ --- \textit{действительной частью} комплексного числа $z$ и обозначается $\Re z$,
	число $b$ --- \textit{мнимой частью} комплексного числа $z$ и обозначается $\Im z$.
	Комплексные числа складываются, вычитаются и перемножаются по тем же законам, по которым производятся операции с
	многочленами, при этом полагается, что $i^2 = -1$.
\end{definition}
\begin{definition}
	\textit{Сопряжённым} к комплексному числу $z$ называют комплексное число $\bar{z} = \Re z - i \Im z$.
\end{definition}
\textit{Везде далее под числом будем понимать комплексное число, если не оговорено иное.}
\begin{problem}
Даны числа $z_1 = 1 + i$, $z_2 = 4 - 3i$.
Найдите
\begin{multienum}{5}
	\item $\Re z_2$;
	\item $\Im z_2$;
	\item $z_1 + z_2$;
	\item $z_1 - z_2$;
	\item $z_1z_2$;
	\item $1/z_1$;
	\item $z_1/z_2$;
	\item $z_1^3$;
	\item $z_1^{2017}$.
\end{multienum}
\end{problem}
\begin{problem}
	Найдите два комплексных числа, сумма которых равна 4, а произведение 5.
\end{problem}
\begin{problem}
\begin{probparts}
	\item Найдите какое-нибудь число $z$,~такое,~что $\Im z \ne 0$, $\Im z^3 = 0$.
	\item Найдите все корни уравнения $z^3 = 1$.
\end{probparts}
\end{problem}
\begin{problem}
	Докажите, что
	\begin{multienum}{3}
		\item \( \widebar{\widebar{z}}  = z\);
		\item \( \Im(\bar z + z) = 0\);
		\item \( \Im(\bar zz) = 0\);
		\item \( \widebar{ z \pm w} = \bar z \pm \bar w\);
		\item \( \widebar{zw} = \bar z \bar w \);
		\item \( \widebar{z/w} = \bar z/ \bar w\).
	\end{multienum}
\end{problem}
\begin{problem}
	Для числа $z = a + bi$ напишите формулы, по которым можно найти противоположное $-z$ и обратное $z^{-1}$ ему числа.
\end{problem}
\begin{problem}
	Найдите общую формулу для частного $(a + bi)/(c + di)$\footnote{Для получения формулы удобно воспользоваться сопряженными числами.}.
\end{problem}
\begin{problem}
	Докажите, что два числа с отличной от нуля мнимой частью являются сопряженными
	тогда и только тогда, когда их сумма и произведение являются действительными числами.
\end{problem}
\begin{definition}
	\textit{Модулем} комплексного числа $z$ называют неотрицательное действительное число $|z| = \sqrt{{(\Re z)}^2 + {(\Im z)}^2}$.
\end{definition}
\begin{problem}
	Верно ли, что в случае действительного числа введенное выше определение модуля не отличается от известного ранее?
\end{problem}
\begin{problem}
	Докажите, что
	\begin{multienum}{3}
		\item $|\bar z| = |z|$;
		\item $\bar zz = |z|^2$;
		\item $|zw| = |z||w|$;
		\item $|z^{-1}| = 1/|z|$;
		\item $|z/w| = |z|/|w|$;
		\item $z^{-1} = \bar z/|z|^2$.
	\end{multienum}
\end{problem}
\begin{definition}
	Комплексное число $z = a + bi$ можно рассматривать как точку координатной плоскости $(a; b)$.
	В этом случае ось абсцисс называется действительной осью, поскольку на ней оказываются действительные числа.
	Множество всех комплексных чисел с нулевой действительной частью (такие числа называются чисто мнимыми) оказывается лежащим на оси ординат,
	которая называется мнимой осью.
	Также любому комплексному числу $z = a + bi$ можно поставить в соответствие вектор $(a; b)$ на координатной плоскости.
	В случае, если на координатной плоскости отмечаются комплексные числа, ее обычно называют комплексной плоскостью.
\end{definition}
\begin{problem}
\begin{probparts}
	\item Отметьте на комплексной плоскости число $z = 2 - 3i$. Где тогда находится число $\bar z$,
	как его можно получить для произвольного числа $z$?
	\item Что с геометрической точки зрения представляет собой $|z|$?
	\item На комплексной плоскости отмечены числа $z$ и $w$.
	Как отметить на ней числа $-\frac 32 z, z + w, z - 2w$?
\end{probparts}
\end{problem}
\begin{problem}
	Верно ли утверждение: <<Операции, совершаемые над комплексными числами, эквивалентны таким же операциям, совершаемым над соответствующими векторами>>?
\end{problem}
\begin{problem}
	Докажите с помощью комплексной плоскости следующие неравенства:
	\begin{multienum}{2}
		\item  $|z_1 + z_2| \le |z_1| + |z_2|$;
		\item  $|z_1 - z_2| \ge |z_1| - |z_2|$.
	\end{multienum}
\end{problem}
\begin{definition}
	\textit{Аргументом} комплексного числа $z$ называется угол на комплексной плоскости,
	отсчитанный против часовой стрелки от положительного направления оси абсцисс до вектора,
	соответствующего числу $z$ и обозначается $\arg z$.
\end{definition}
\begin{problem}
\begin{probparts}
	\item Найдите число $z$, если $|z| = 2$, $\arg z = \frac{2\pi}3$.
	\item Найдите модуль и аргумент числа $z = 2 - 3i$.
\end{probparts}
\end{problem}
\begin{problem}[(Тригонометрическая форма записи комплексного числа)]
	Докажите, что любое комплексное число $z$ может быть записано в виде
	\[
		z = r(\cos \phi + i \sin \phi),
	\]
	где $r = |z|$, $\phi = \arg z$.
\end{problem}
\begin{problem}
\begin{probparts}
	\item Как выражаются модуль и аргумент произведения двух комплексных чисел через их модули и аргументы?
	\item Тот же вопрос для частного.
\end{probparts}
\end{problem}
\begin{problem}[(Формула Муавра)]
	Пусть $r = |z|$, $\phi = \arg z$.
	Докажите, что тогда
	\[
		z^n = r^n (\cos{n\phi} + i \sin{n\phi}).
	\]
\end{problem}
\begin{problem}
	Вычислите ${\left ( \frac 3 2 - i \frac{\sqrt{27}}2 \right )}^{33}$.
\end{problem}
\begin{problem}
\begin{probparts}
	\item Выведите с помощью формулы Муавра формулы для синуса и косинуса тройного и четверного угла.
	\item Выразите $\sin{nx}$ и $\cos{nx}$ в виде многочленов от $\sin x$ и $\cos x$.
\end{probparts}
\end{problem}
\begin{problem}
	Вычислите суммы $\cos \alpha + \cos{2\alpha} + \ldots + \cos{n\alpha}$ и $\sin \alpha + \sin{2\alpha} + \ldots + \sin{n\alpha}$.
\end{problem}
\end{document}
