\documentclass[a4paper, 12pt, num=2729, date=01.09.2020]{listok}
\usepackage{bm}
\usepackage{halloweenmath,fontawesome}
%\usepackage{scalerel}
%\usepackage{tikz}
%\usepackage{epigraph}
%\usepackage{multicol}
%\usetikzlibrary{matrix}
%\usetikzlibrary{calc}
%\usetikzlibrary{arrows}

\DeclareMathOperator{\id}{id}

%\newcommand*{\hm}[1]{#1\nobreak\discretionary{}{\hbox{$\mathsurround=0pt #1$}}{}} %перед знаком для его дублирования при переносе формулы

\begin{document}

\makeatletter{}
\renewcommand{\listok@num}{\faRocket}
\makeatother{}

\title{Speedrun. Топология прямой}
\maketitle{}

\setlength{\abovedisplayskip}{5pt plus 2pt minus 2pt}
\setlength{\belowdisplayskip}{5pt plus 2pt minus 2pt}

\begin{center}
    \textit{Символом $\mathghost$ обозначены задачи, при решении которых могут возникать непреодолимые трудности.
    Если Вы долго думаете над задачей без каких-то продвижений, стоит обратиться к ближайшему преподавателю математического анализа.}
\end{center}

\begin{definition}
    Число $b \in \R$ называется \textit{верхней гранью} для данного множества действительных чисел $M \subset \R$,
    если $\forall{x \in M} b \ge x$. Множество $M \subset \R$ называется \textit{ограниченным сверху}, если у него есть верхняя грань.
    Аналогично определяются \textit{нижняя грань} и \textit{ограниченность снизу}.
    Множества, ограниченные одновременно и сверху, и снизу, называются просто \textit{ограниченными}.
\end{definition}

\begin{problem}
    Не употребляя отрицаний, дайте определения
    \begin{probparts}
        \item числа, не являющегося нижней гранью данного множества $M \subset \R$;
        \item множества, неограниченного сверху;
        \item неограниченного множества.
    \end{probparts}
    Запишите их с помощью кванторов.
\end{problem}

\begin{definition}
    Число $m \in M$, для которого $\forall{x \in M} m \ge x$, называется \textit{максимальным элементом} множества $M \subset \R$.
    Максимальный элемент в множестве нижних граней данного множества $M$ называется \textit{точной нижней гранью} (сокращённо: тнг) и обозначается $\inf M$ (от латинского <<infimum>>).
\end{definition}

\begin{problem}
    Дайте определение минимального элемента и точной верхней грани (сокращённо: твг) данного множества $M \subset \R$
    (она обозначается $\sup M$ --- от латинского <<supremum>>).
\end{problem}

\begin{problem}
    Каждое ли ограниченное сверху подмножество в $\R$ имеет максимальный элемент?
\end{problem}

\begin{problem}
    Вычислите твг множеств:
    \begin{probparts}
        \item $\{0{,}3; 0{,}33; 0{,}333; 0{,}3333 ; 0{,}33333 ; \ldots \}$;
        \item сумм $1 + q + q^2 + \cdots + q^m$ с фиксированным $0< q< 1$ и любым $m \in \N$.
    \end{probparts}
\end{problem}

\begin{problem}
\begin{probparts}
    \item Докажите, что твг $b = \sup M$ ограниченного сверху множества $M \subset \R$ обладает свойством \textit{предельности}:
        для любого $\epsilon > 0$ найдётся такое $x \in M$, что $b - \epsilon \le x \le b$.
    \item Докажите, что это свойство является достаточным, то есть что любая верхняя грань, обладающая этим свойством, будет точной.
\end{probparts}
\end{problem}

Что такое число? Понятно, что как-то можно разобраться с тем, как устроено множество натуральных чисел $\N$. Оно состоит из элементов, которые следуют друг за другом по цепочке, начиная с $1$. Несложно определить операции: например, чтобы прибавить к числу $n \in \N$ число $m \in \N$, надо $m$ раз взять следующее число после $n$. Целые числа $\Z$ легко строятся из натуральных, а рациональные $\Q$ --- из целых и натуральных как множество дробей. Вопрос о том, из чего состоит множество действительных чисел $\R$, оказывается далеко не столь прост, и можно отвечать на него разными способами. 

Часто действительные числа воспринимают как точки на геометрической прямой, но не все при этом понимают, как определяется прямая в элементарной геометрии (как?). 

Другое привычное определение множества действительных чисел --- бесконечные десятичные дроби, среди которых надо отождествить бесконечные десятичные дроби с бесконечным числом девяток на конце с соответствующими конечными десятичными дробями. При таком определении действительных чисел не так легко, как кажется на первый взгляд, определять операции: например, как сложить две бесконечные десятичные дроби, нельзя же складывать в столбик, начиная с конца, если конца нет? Или, хуже, как перемножить? И почему после того, как мы дадим какие-то определения операций, будут выполняться привычные нам из начальной школы свойства вроде дистрибутивности $a(b + c) = ab + ac$ и ассоциативностей $(ab)c = a(bc), \ (a + b) + c = a + (b + c)$? 

Есть ещё определение действительных чисел как так называемых Дедекиндовых сечений --- число $x \in \R$ определяется как разбиение множества рациональных чисел на две такие непересекающиеся части $\Q = A \sqcup B$, что любое число из $A$ меньше любого числа из $B$ (+ некоторое техническое условие). По сути множество $A$ состоит из рациональных чисел, не превосходящих $x$, а $B$ --- из чисел, больших $x$. С такими громоздкими конструкциями работать довольно непривычно, но зато несложно определяются операции сложения и умножения.

Геометрическая прямая, бесконечные десятичные дроби и Дедекиндовы сечения --- разные <<модели>> одного и того же понятия. Мы не будем углубляться в определение операций и доказательства их простых свойств в разных моделях, можно ими пользоваться без доказательства. Однако некоторые свойства действительных чисел, которые следуют из конкретных моделей выше, нельзя вывести из свойств сложения и умножения. Одно из таких свойств мы примем в качестве аксиомы. 

\smallskip

\noindent\textbf{Аксиома полноты}: у любого ограниченного сверху множества $M \subset \R$ существует твг $\sup M \in \R$. 

\begin{problem}
    Объясните, почему при замене $\R$ на $\Q$ аксиома полноты перестаёт выполняться. 
\end{problem}

\begin{problem}
    Докажите, что каждое ограниченное снизу подмножество $\R$ имеет точную нижнюю грань. 
\end{problem}

\begin{problem}[ (лемма о вложенных отрезках) $\mathghost$]\label{subsec}
    Имеется последовательность отрезков, каждый из которых содержится в предыдущем.
    Докажите, что пересечение всех этих отрезков непусто.\footnote{На самом деле эта лемма равносильна аксиоме полноты и можно взять в качестве аксиомы её.}
\end{problem}

\begin{problem}[ (лемма о стягивающихся отрезках) $\mathghost$]\label{subsec}
    Пусть про последовательность отрезков из предыдущей задачи известно,
    что последовательность их длин является бесконечно малой.
    Докажите, что пересечение этих отрезков состоит из одной точки.
\end{problem}

\begin{problem}
    Верна ли лемма для вложенных интервалов? 
\end{problem}

\begin{problem}[ (лемма о конечном покрытии)$\mathghost$]\label{coversec}
    Докажите, что в любом покрытии отрезка интервалами найдётся конечный набор интервалов, покрывающий весь отрезок. \\
\end{problem}
\begin{note}
    Действуйте от противного и поделите отрезок пополам.
\end{note}

\begin{problem}
    Верна ли лемма о конечном покрытии для интервала? 
\end{problem}

\begin{definition}
    Для любого положительного числа $\epsilon$ будем называть \textit{$\epsilon$-окрестностью} точки $x_0 \in \R$ интервал
    $
        D_{\epsilon}(x_0) := \left ( x_0 - \epsilon, x_0 + \epsilon \right )
        = \left \{ x \in \R \mid |x - x_0| < \epsilon \right \}
    $
    длины $2 \epsilon$ с центром в $x_0$.
\end{definition}

\begin{definition}
    Точка $a \in \R$ называется \textit{внутренней точкой} множества $M \subset \R$, если у неё есть $\epsilon$-окрестность, целиком содержащаяся в $M$.
\end{definition}

\begin{definition}
    Множество называется \textit{открытым}, если все его точки внутренние. Иначе говоря, $U \subset \R$ открыто, если
    $\forall{u \in U} \;\; \exists{\epsilon > 0} \;\; D_{\epsilon}(u) \subset U$.
    Пустое множество тоже по определению считается открытым.
\end{definition}

\begin{problem}
    Убедитесь, что промежутки $(-\infty, a)$, $(a, +\infty)$ и $(a, b)$ открыты.
\end{problem}

\begin{problem}
    Докажите, что объединение любого набора открытых множеств открыто.
\end{problem}

\begin{problem}
\begin{probparts}
    \item Докажите, что пересечение конечного набора открытых множеств открыто.
    \item Так ли это для пересечений бесконечных наборов.
\end{probparts}
\end{problem}

\begin{problem}[ (лемма о конечном покрытии-2)$\mathghost$]
    Докажите, что в любом покрытии отрезка открытыми множествами найдётся конечный набор этих множеств, покрывающий весь отрезок.
\end{problem}
\begin{note}
    Сведите к интервалам или докажите так же.
\end{note}

%\begin{problem}%!!!!!!
%    Можно ли разбить интервал в объединение двух непересекающихся открытых множеств? \; 
%    \emph{Указание}: Рассмотрите $M = \{x \in (a;b) \mid (a;x) \text{ из одного множества}\}$.????
%\end{problem}

\begin{definition}
    Точка $x_0 \in \R$ называется \textit{предельной} точкой множества $M \subset \R$,
    если любая её $\epsilon$-окрестность $D_{\epsilon}(x_0)$ содержит какую-нибудь точку $x \in M$, отличную от $x_0$.
\end{definition}

\begin{example*}
    Рассмотрим множество $M = \{\frac1n \mid n \in \N\}$. Точка $0$ --- предельная точка $M$, так как для любого $\varepsilon > 0$ существует $n > \frac1\epsilon$, и для этого $n$ выполнено $0 < \frac1n < \varepsilon$, то есть точка $\frac1n \in M$ лежит в $D_{\epsilon}(0)$ и не совпадает с $0$. 
\end{example*}

\begin{problem}
    Постройте бесконечное множество $M \subset \R$, множество предельных точек которого:
    \begin{probparts}
        \item пусто;
        \item состоит из двух точек;
        \item совпадает с $\Z \subset \R$.
    \end{probparts}
\end{problem}

\begin{problem}[$\mathghost$]
    Докажите, что бесконечное ограниченное множество имеет предельные точки.
\end{problem}
\begin{note}
    Поделите отрезок пополам.
\end{note}

\begin{definition}
    Множество $Z \subset \R$ называется \textit{замкнутым}, если оно содержит все свои предельные точки (в частности, если их у него вообще нет).
\end{definition}

\begin{problem}
    Докажите, что отрезок и прямая замкнуты, а интервал и луч --- нет.
\end{problem}

\begin{problem}
    Докажите, что множество $Z \subset \R$ замкнуто тогда и только тогда, когда его дополнение $\R \setminus Z$ открыто.\footnote{Очень полезная задача --- помогает сводить замкнутые множества к открытым\ldots}
\end{problem}

\begin{problem}
    Докажите, что пересечение любого набора замкнутых множеств замкнуто.
\end{problem}

\begin{problem}
\begin{probparts}
    \item Докажите, что объединение конечного набора замкнутых множеств замкнуто.
    \item Так ли это для бесконечных наборов?
\end{probparts}
\end{problem}

%\begin{problem}
%    Разбивается ли отрезок в объединение двух непересекающихся непустых замкнутых множеств?
%\end{problem}

\begin{definition}
    Непустые ограниченные замкнутые множества называются \textit{компактами}.
\end{definition}

\begin{example*}
    Любой отрезок является компактом. 
\end{example*}

\begin{problem}[ (лемма о конечном покрытии-3)$\mathghost$]
    Докажите, что непустое $K \subset \R$ компакт тогда и только тогда, когда любое его покрытие открытыми множествами содержит конечное подпокрытие. \\
\end{problem}
\begin{note}
    $\Rightarrow$ Компакт $K \subseteq [a;b]$ (почему?) Покройте $[a;b]$ и воспользуйтесь\ldots. \\
    $\Leftarrow$ Если $K$ не ограничено, рассмотрите покрытие интервалами длины 1. 
    Если $K$ не замкнуто, то у него есть предельная точка $x_0 \notin K$: рассмотрите $\{x \in \R \mid |x - x_0| > 1/n\}$. 
    %\footnote{%    Подсказка: Сравните эти две задачи с задачами~\ref{subsec} и~\ref{coversec}.}
\end{note}

\begin{problem}[ (лемма о вложенных компактах)]
    Докажите, что любая последовательность вложенных компактов $K_1 \supset K_2 \supset K_3 \supset \ldots$
    имеет непустое пересечение: $\bigcap K_n \ne \emptyset$. \\
\end{problem}
\begin{note}
    \emph{Указание}: удобно действовать от противного. Не забыв про ограниченность $K_1$, рассмотрите дополнения к $K_i$ --- что они покрывают?
\end{note}

\end{document}
