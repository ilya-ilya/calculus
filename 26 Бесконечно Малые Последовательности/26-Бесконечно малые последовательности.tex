\documentclass[a4paper, 12pt, num=26]{listok}
\usepackage{bm}
%\usepackage{scalerel}
%\usepackage{tikz}
%\usepackage{epigraph}
%\usepackage{multicol}
%\usetikzlibrary{matrix}
%\usetikzlibrary{calc}
%\usetikzlibrary{arrows}

\DeclareMathOperator{\id}{id}

%\newcommand*{\hm}[1]{#1\nobreak\discretionary{}{\hbox{$\mathsurround=0pt #1$}}{}} %перед знаком для его дублирования при переносе формулы

\begin{document}
\title{Бесконечно малые последовательности}
\maketitle{}
\begin{definition}
	Последовательность $\{\alpha_n\}$ называется \textit{бесконечно малой}, если для произвольного положительного числа $\epsilon$
	найдётся такое натуральное число $N$, что при любом натуральном $n > N$ будет верно неравенство $|\alpha_n| < \epsilon$.
\end{definition}
\begin{center}
	\textit{В задачах ниже при доказательстве надо явно указывать сколемовскую функцию.}
\end{center}
\begin{problem}
	Запишите в кванторах определение бесконечно малой последовательности и последовательности, не являющейся бесконечно малой.
	Запишите их так, чтобы кванторы существования шли в начале формулы.
\end{problem}
\begin{problem}
	Для последовательности $\{\alpha_n\}$ укажите какой-нибудь номер $N$, начиная с которого для всех членов последовательности верно неравенство
	$|\alpha_n| < \epsilon$, если
	\begin{multienum}{3}
		\item $\alpha_n = \cfrac1n$;
		\item $\alpha_n = \cfrac{{(-1)}^n \cos n}{n^3}$;
		\item $\alpha_n = {0{,}99}^n$;
		\item $\alpha_n = \cfrac{2^n}{n!}$;
		\item $\alpha_n = \cfrac{2n+3}{n^2+2n-1}$;
		\item $\alpha_n = \sqrt{n + 1} - \sqrt n$.
	\end{multienum}
\end{problem}
\begin{problem}
	Верно ли, что
	\begin{probparts}
		\item сумма;
		\item разность;
		\item произведение;
		\item отношение
	\end{probparts}
	бесконечно малых последовательностей также является бесконечно малой последовательностью.
\end{problem}
\begin{problem}
	Пусть последовательности $\{\alpha_n\}$ и $\{\beta_n\}$ являются бесконечно малыми,
	а последовательность $\{\gamma_n\}$ такова, что $\alpha_n \le \gamma_n \le \beta_n$.
	Докажите, что тогда последовательность $\{\gamma_n\}$ также является бесконечно малой.
\end{problem}
\begin{problem}
	Верно ли, что последовательность $\{\alpha_n\}$ с отличными от нуля членами является бесконечно малой тогда и только тогда,
	когда последовательность $\{\frac1{\alpha_n}\}$ является бесконечно большой?
\end{problem}
\begin{problem}
\begin{probparts}
	\item Верно ли, что последовательность $\{\alpha_n\}$ положительными членами является бесконечно малой тогда и только тогда,
	когда последовательность $\{\alpha_n^2\}$ является бесконечно малой?
	\item Верно ли аналогичное утверждение для последовательностей $\{\alpha_n\}$ и $\{\sqrt[3]{\alpha_n}\}$?
\end{probparts}
\end{problem}
\begin{problem}
	Последовательности $\{\alpha_n\}$ и $\{\beta_n\}$ бесконечно малые, а последовательность $\{\gamma_n\}$ такова,
	что $\gamma_{2n-1} = \alpha_n$, $\gamma_{2n} = \beta_n$ при любом натуральном $n$.
	Является ли последовательность $\{\gamma_n\}$ бесконечно малой?
\end{problem}
\begin{problem}
	Одна последовательность бесконечно малая, а другая ограниченная.
	Что можно сказать о
	\begin{probparts}
		\item сумме;
		\item произведении;
		\item отношении
	\end{probparts}
	этих последовательностей?
\end{problem}
\begin{problem}
	Решите предыдущую задачу в случае, если одна из последовательностей бесконечно малая, а другая бесконечно большая.
\end{problem}
\begin{problem}
	Докажите, пользуясь предыдущими задачами, что следующие последовательности являются бесконечно малыми:
	\begin{multienum}{4}
		\item $\cfrac{n^5 + 3}{n^{10}}$;
		\item $\cfrac{3n^6 + 2n^4 - n}{n^9 + 7n^5 - 5n^2 - 2}$;
		\item $\sqrt{\cfrac{|\sin{3n} + \cos{7n}|}{2n^2 + 3n}}$;
		\item $\cfrac{3^n + 4^n}{2^n + 5^n}$;
	\end{multienum}
\end{problem}
\begin{problem}
	Любую ли последовательность можно представить в виде отношения двух бесконечно малых последовательностей?
\end{problem}
\begin{problem}
	По последовательности $\{\alpha_n\}$ построили последовательность $\{\beta_n\}$ так, что
	$\beta_n = \alpha_{n + 1} - \frac{\alpha_n}2$ при любом натурального $n$.
	Докажите, что если последовательность $\{\beta_n\}$ оказалась бесконечно малой,
	то и последовательность $\{\alpha_n\}$ также является бесконечно малой.
\end{problem}
\end{document}
