\documentclass[a4paper, 12pt, num=2426, date=01.09.2020]{listok}
\usepackage{bm}
\usepackage{halloweenmath,fontawesome}
%\usepackage{scalerel}
%\usepackage{tikz}
%\usepackage{epigraph}
%\usepackage{multicol}
%\usetikzlibrary{matrix}
%\usetikzlibrary{calc}
%\usetikzlibrary{arrows}

\DeclareMathOperator{\id}{id}

%\newcommand*{\hm}[1]{#1\nobreak\discretionary{}{\hbox{$\mathsurround=0pt #1$}}{}} %перед знаком для его дублирования при переносе формулы

\begin{document}

\makeatletter{}
\renewcommand{\listok@num}{\faCar}%{\faRocket}
\makeatother{}

\title{Speedrun. Последовательности}
\maketitle{}

\setlength{\abovedisplayskip}{5pt plus 2pt minus 2pt}
\setlength{\belowdisplayskip}{5pt plus 2pt minus 2pt}

\begin{center}
    \textit{Символом $\mathghost$ обозначены задачи, при решении которых могут возникать непреодолимые трудности.
    Если Вы долго думаете над задачей без каких-то продвижений, стоит обратиться к ближайшему преподавателю математического анализа.}
\end{center}

\begin{definition}
    Пусть имеется некоторое непустое множество $U$.
    Любая функция $a \colon \N \to U$ называется \textit{последовательностью элементов множества $U$}.
    Иначе говоря, каждому натуральному числу $n \in \N$ ставится в соответствие элемент последовательности $a(n) \in U$, который обычно обозначают через $a_n$.
    Саму последовательность обозначают $\{a_n\}$ или ${\{a_n\}}_{n = 1}^{\infty}$.
\end{definition}

\begin{definition}
    \textit{Числовой последовательностью} называют последовательность элементов какого-нибудь фиксированного множества чисел ($\N, \Z, \Q, \R, \Cx$).
    Обычно, если иное специально не оговорено, говоря о числовых последовательностях, мы будем иметь в виду $\R$-последовательности.
\end{definition}

\begin{example*}
    \begin{enumerate}
        \item Последовательность полных графов разного размера $\{K_n\}$.
        \item Последовательность $2, 3, 4, 5\ldots$, в которой значение каждого элемента на единицу больше его номера,
            можно записать как $\{n + 1\}$ или $\{a_n\}$, где $a_n = n + 1$.
        \item Последовательность $1, 2, 1, 4, 1, 6, 1, 8\ldots$ можно задать формулой
            $b_n = \begin{cases} 1, \text{если $n$ нечётное;}\\ n, \text{если $n$ чётное.} \end{cases}$
    \end{enumerate}
\end{example*}

\begin{definition}
    Последовательность $\{a_n\}$ называется \textit{ограниченной сверху}, если найдётся такое число $C$,
    что при всех натуральных $n$ будет выполнено неравенство $a_n < C$.
\end{definition}

\begin{problem}
    Дайте определение последовательности, ограниченной снизу.
\end{problem}

\begin{problem}
    Приведите пример последовательности
    \begin{probparts}
        \item ограниченной сверху, но не ограниченной снизу;
        \item не ограниченной ни сверху, ни снизу.
    \end{probparts}
\end{problem}

\begin{definition}
    Последовательность называется \textit{ограниченной}, если она ограничена и сверху, и снизу.
\end{definition}

\begin{problem}
    Докажите, что последовательность $\{a_n\}$ ограничена тогда и только тогда,
    когда для некоторого числа $C > 0$ при всех натуральных $n$ выполнено неравенство $|a_n| < C$.
\end{problem}

\begin{problem}
    Приведите пример ограниченной вещественной последовательности, у которой
    \begin{probparts}
        \item есть и наибольший, и наименьший член;
        \item есть наибольший, но нет наименьшего члена;
        \item есть наименьший, но нет наибольшего члена;
        \item нет ни наименьшего, ни наибольшего члена.
    \end{probparts}
\end{problem}

\begin{example*}
\begin{enumerate}
    \item Исследуем на ограниченность последовательность $a_n = \frac{n^2}{2^n}$.
        Так как $a_n > 0$, то она ограничена снизу.
        Докажем теперь, что $a_n \le a_4$ для всех $n$ --- из этого будет следовать ограниченность сверху.
        Сравним $a_n$ и $a_{n+1}$: \;
        $\frac{n^2}{2^n} \vee \frac{{(n+1)}^2}{2^{n+1}}$, \; $2n^2 \vee {(n+1)}^2$, \; $n^2 - 2n - 1 \vee 0$.
        Так как корни квадратного уравнения в левой части равны $1 \pm \sqrt{2}$ и $n \in N$, получим,
        что $a_n < a_{n+1}$при $n \le 3$ и $a_n \ge a_{n+1}$ при $n \ge 4$,
        то есть $a_4$ больше всех остальных элементов последовательности.

    \item Один из способов доказать, что последовательность не ограничена сверху ---
        оценить её снизу чем-то более простым, не ограниченным сверху.
        Так, докажем, что последовательность $a_n = n + \sin n$ не ограничена сверху.
        Ясно, что для любого $n$ выполнено $n - 1 \le a_n$.
        Если бы для всех $n$ было $a_n < C$, то и $n - 1 \le a_n < C$, но это не так, например,
        для $n = \lceil C\rceil + 2$.
\end{enumerate}
\end{example*}

\begin{problem}[$\mathghost$]\label{exfirst}
    Исследуйте на ограниченность следующие последовательности, а также изобразите их на координатной плоскости:
    \begin{probparts}
        \item $a_n = \frac{100^n}{n!}$;
        \item $a_n = 1{,}01^n$.
    \end{probparts}
\end{problem}
\begin{note}
\begin{probparts}
    \item сравните $a_n$ и $a_{n+1}$;
    \item ${(1 + 0{,}01)}^n = \ldots + \ldots + $ что-то неотрицательное.
\end{probparts}
\end{note}
\begin{definition}
    Суммой последовательностей $\{a_n\}$ и $\{b_n\}$ называется такая последовательность $\{c_n\}$,
    что $c_n = a_n + b_n$ при всех $n \in \N$.
    Аналогичным образом определяют разность, произведение, отношение двух последовательностей.
\end{definition}

\begin{problem}[$\mathghost$]
    Приведите пример двух последовательностей, хотя бы одна из которых не ограниченная, но
    \begin{probparts}
        \item сумма;
        \item произведение
    \end{probparts}
    которых --- ограниченная последовательность.
\end{problem}

\begin{example*}
    Докажем, что сумма ограниченных последовательностей $\{a_n\}$ и $\{b_n\}$
    является ограниченной последовательностью.
    Так как $\{a_n\}$ ограничена, то существует такое число $A > 0$,
    что для всех натуральных $n$ выполнено $|a_n| < A$, или $-A < a_n < A$.
    Аналогично существует такое число $B > 0$, что для всех натуральных $n$ выполнено $-B < b_n < B$.
    Тогда для всех натуральных $n$, сложив неравенства,
    получим $a_n + b_n < A + B$ и $a_n + b_n > -A-B$, то есть последовательность $\{a_n + b_n\}$ ограничена.
\end{example*}

\begin{problem}
    Верно ли, что
    \begin{probparts}
        \item разность;
        \item произведение;
        \item отношение
    \end{probparts}
    ограниченных последовательностей --- ограниченная последовательность?
\end{problem}

\begin{problem}[$\mathghost$]
    Являются ли ограниченными последовательности:
    \begin{probparts}
        \item\label{lista} $a_n = \sum\limits_{i = 1}^n \frac1{2^i}$;
        \item\label{listb} $a_n = \sum\limits_{i = 1}^n \frac1i$;
        \item\label{listv} $a_n = \sum\limits_{i = 1}^n \frac1{i(i + 1)}$;
        \item\label{listg} $a_n = \sum\limits_{i = 1}^n \frac1{i^2}$;
        \item\label{listd} $a_n = \sum\limits_{i = 1}^n \frac1{i!}$.
    \end{probparts}
\end{problem}
\begin{note}
    \ref{lista},~\ref{listv} --- найдите формулу;~\ref{listb} оцените слагаемые снизу степенями $2$;
    \ref{listg},~\ref{listd}~$\Leftarrow$~\ref{listv}.
\end{note}
%\begin{problem}[$\mathghost$]\label{exsecond}
%    Докажите, что следующие последовательности являются ограниченными: \\
%        \item $a_1 = 1, a_{n + 1} = \sqrt{2 + a_n}$;
%        (б$^*$) $a_n = {(1 + \frac1n)}^n$.\\
%    Указания: (а) докажите по индукции (что?); (б) сравните с $b_n = {(1 + \frac1n)}^{n+1}$.
%\end{problem}

\begin{definition}
    Пусть имеется некоторое множество $U$ (\textit{универсум})
    и некоторое утверждение (\textit{предикат}) $A$ про его элементы.
    То есть для каждого $a \in U$ мы знаем либо, что $A(a)$ верно, либо, что неверно.
    Для построения стандартных математических суждений принято использовать \textit{кванторы всеобщности и существования} следующим образом
    \begin{align*}
        \forall{a \in U} \; A(a) &\quad\text{читают как <<для любого $a \in U$ верно $A(a)$>>},\\
        \exists{a \in U} \; A(a) &\quad\text{читают как <<существует такой $a \in U$, что верно $A(a)$>>}.
    \end{align*}
    Иногда также выделяют квантор <<существует единственный>> $\existssym!$.
\end{definition}

\begin{example*}
\begin{enumerate}
    \item Рассмотрим суждение <<$\forall{x \in \N} \exists{y \in \N} y = x + 1$>> = <<для любого натурального числа $x$ существует на единицу большее его число $y$>>.
    Оно верно. Заметим, что если поменять в нём местами два квантора, то оно перестанет быть верным:
    \mbox{<<$\exists{y \in \N} \forall{x \in \N} y = x + 1$>>} = <<есть число $y$, на единицу большее любого другого числа~$x$>>.
    \item <<$\exists{x \in \R} \forall{y \in \R} x \le |y|$>> = <<есть число $x$, не превосходящее модуля любого другого числа $|y|$>>. Действительно, в качестве такого $x$ можно взять $0$, и для любого $y$ будет выполнено $0 \le |y|$.
    \item $\forall{x \in \N} \exists{y \in \R} x < y$.
    \item $\forall{x \in \N} \exists{y \in \R} y < x$.
\end{enumerate}
\end{example*}

\begin{problem}
    В каких утверждениях 2.--4. из примера можно поменять местами кванторы всеобщности и существования, а в каких нельзя? Почему?
\end{problem}

\begin{problem}
    Запишите с помощью кванторов определения ограниченной снизу, ограниченной, возрастающей, убывающей, невозрастающей, неубывающей последовательностей.
\end{problem}

\begin{definition}
    Последовательность называется \textit{монотонной}, если она является неубывающей либо невозрастающей.
    Последовательность называется \textit{строго монотонной}, если она является либо возрастающей, либо убывающей.
    Очевидно, что строго монотонная последовательность является монотонной.
\end{definition}

\begin{problem}
    Сформулируйте, не используя отрицания, определение последовательности, которая
    \begin{probparts}
        \item не является возрастающей;
        \item не является ограниченной сверху;
        \item не является монотонной.
    \end{probparts}
    Запишите их с помощью кванторов.
\end{problem}

\begin{problem}
    Про каждую из последовательностей задачи~\ref{exfirst} выясните,
    является ли она монотонной, и найдите, если это возможно, ее наибольший и наименьший члены.
\end{problem}

\pagebreak

\begin{definition}
    Последовательность называется \textit{неограниченной}, если она не является ограниченной.
\end{definition}

\begin{definition}
    Последовательность $\{a_n\}$ называется \textit{бесконечно большой}, если для любого числа $C > 0$ найдётся такое натуральное число $N$,
    что при всех $n > N$ выполнено неравенство $|a_n| > C$.
\end{definition}

\begin{problem}
    Сформулируйте, не используя отрицания, определения неограниченной и бесконечно большой последовательностей с помощью кванторов.
    Верно ли, что любая бесконечно большая последовательность является неограниченной? А наоборот?
    Рассмотрим множество неограниченных и множество бесконечно больших последовательностей. Пересекаются ли они?
    Является ли одно из них подмножеством другого?
\end{problem}

\begin{problem}
    Сформулируйте, не используя отрицания, определение последовательности, которая
    \begin{probparts}
        \item не является неограниченной;
        \item не является бесконечно большой.
    \end{probparts}
    Запишите их с помощью кванторов.
\end{problem}

\begin{problem}
    Последовательность $\{a_n\}$ бесконечно большая. Верно ли, что она монотонная?
\end{problem}

\begin{problem}
    Верно ли, что
    \begin{probparts}
        \item сумма;
        \item разность;
        \item произведение;
        \item отношение бесконечно больших последовательностей --- бесконечно большая последовательность?
        \item Докажите, что сумма, разность, произведение и отношение неограниченных последовательностей --- не обязательно неограниченная последовательность.
    \end{probparts}
\end{problem}

\begin{definition}
    Пусть ${\{n_i\}}_{i=1}^{\infty}$ --- возрастающая последовательность натуральных чисел.
    Последовательность ${\{b_i\}}_{i=1}^{\infty}$, где $b_i = a_{n_i}$, называется \textit{подпоследовательностью} последовательности ${\{a_n\}}_{n=1}^{\infty}$.
\end{definition}

\begin{example*}
    Для $\{a_n\} = 0, 3, 0, 6, 0, 9, 0,\ldots$ и $\{n_i\} = \{2i\} = 2, 4, 6, \ldots$ получаем подпоследовательность $\{a_{n_i}\} = a_2, a_4, a_6\ldots = 3, 6, 9\ldots = \{3i\}$.
\end{example*}

\begin{example*}
    Докажем, что любая подпоследовательность монотонной последовательности монотонна.
    Так как для любого $i \in \N$ выполнено $n_i < n_{i+1}$, имеем цепочку последовательных натуральных чисел $n_i, n_i + 1, n_i + 2, \ldots, n_{i+1}$.
    Так как $\{a_n\}$ монотонна, её элементы с соответствующими номерами удовлетворяют неравенству $a_{n_i} \le a_{n_i + 1} \le \ldots \le a_{n_{i+1}}$ (или тому же с $\ge$).
    Отсюда $a_{n_i} \le a_{n_{i+1}}$ для всех $i$ (или $\ge$), то есть подпоследовательность ${\{a_{n_i}\}}_{i = 1}^{\infty}$ монотонна.
\end{example*}

\begin{problem}
\begin{probparts}
    \item Докажите, что любая подпоследовательность ограниченной последовательности ограничена.
    \item Докажите, что для неограниченных это не верно.
    \item Верно ли для бесконечно больших?
\end{probparts}
\end{problem}

\noindent Из предыдущих задачи и примера видно, что некоторые свойства последовательностей не теряются при переходе к подпоследовательностям. Кроме того, перейдя к подпоследовательности, какие-то полезные свойства можно приобрести.  Сформулируем без доказательства два соответствующих утверждения.

\emph{Факт 1}: Любая последовательность содержит монотонную подпоследовательность.

\emph{Факт 2}: Любая неограниченная последовательность содержит бесконечно большую подпоследовательность.

\begin{problem}
\begin{probparts}
    \item Придумайте какую-нибудь последовательность, не монотонную начиная с любого номера, и найдите в ней монотонную подпоследовательность (факт 1).
    \item Возьмите какую-нибудь неограниченную последовательность, которая не является бесконечно большой, и найдите в ней бесконечно большую подпоследовательность (факт 2).
\end{probparts}
\end{problem}

\begin{problem}
\begin{probparts}
    \item Докажите, что для любой ограниченной последовательности существует отрезок длины 1, в котором находится бесконечно много членов этой последовательности.
    \item Докажите, что если для некоторой последовательности такого отрезка длины 1 найти нельзя, то эта последовательность  бесконечно большая.
\end{probparts}
\end{problem}

\begin{definition}
    Последовательность $\{\alpha_n\}$ называется \textit{бесконечно малой}, если для произвольного числа $\epsilon > 0$
    найдётся такое натуральное число $N$, что при любом натуральном $n > N$ будет верно неравенство $|\alpha_n| < \epsilon$.
\end{definition}

\begin{problem}
    Запишите в кванторах определение бесконечно малой последовательности и последовательности, не являющейся бесконечно малой.
\end{problem}

\begin{problem}
    Для последовательности $\{\alpha_n\}$ и произвольного $\epsilon > 0$ укажите какой-нибудь номер $N$, начиная с которого для всех членов последовательности верно неравенство $|\alpha_n| < \epsilon$, если
    \begin{probparts}
        \item $\alpha_n = \frac1n$;
        \item $\alpha_n = \frac{{(-1)}^n \cos n}{n^3}$;
        \item $\mathghost$ $\alpha_n = \sqrt{n + 1} - \sqrt n$.
    \end{probparts}
\end{problem}

\begin{example*}
    Докажем, что сумма двух бесконечно малых последовательностей $\{\alpha_n\}$ и $\{\beta_n\}$ является бесконечно малой последовательностью. Возьмём проивольное $\epsilon > 0$. По определению для $\frac\epsilon2 > 0$ существует такой номер $N \in \N$, что для всех $n > N$ выполнено $|\alpha_n| < \frac\epsilon2$, то есть $-\frac\epsilon2 < \alpha_n < \frac\epsilon2$. Для того же $\epsilon$ есть номер $M \in \N$, что для всех $n > M$ выполнено $|\beta_n| < \frac\epsilon2$, то есть $-\frac\epsilon2 < \beta_n < \frac\epsilon2$ (номер $M$ вообще говоря другой!). Возьмём номер $K = \max(N, M)$. При любом $n > K$ выполнены оба двойных неравенства, и, складывая отдельно правые и левые, получим \[-\varepsilon = -\frac\epsilon2 - \frac\varepsilon2 < \alpha_n + \beta_n < \frac\epsilon2 + \frac\varepsilon2 = \varepsilon.\]
    Итого мы для любого $\varepsilon > 0$ нашли номер $K \in \N$, такой что для всех $n > k$ выполнено $|\alpha_n + \beta_n| < \varepsilon$, что и означает, что $\{\alpha_n + \beta_n\}$ бесконечно малая.
\end{example*}

\begin{problem}
    Верно ли, что
    \begin{probparts}
        \item разность;
        \item произведение;
        \item отношение бесконечно малых последовательностей --- бесконечно малая последовательность?
    \end{probparts}
\end{problem}

\begin{problem}
    Пусть последовательности $\{\alpha_n\}$ и $\{\beta_n\}$ являются бесконечно малыми,
    а последовательность $\{\gamma_n\}$ такова, что $\alpha_n \le \gamma_n \le \beta_n$.
    Докажите, что тогда последовательность $\{\gamma_n\}$ также является бесконечно малой.
\end{problem}

\begin{problem}
    Докажите, что последовательность $\{\alpha_n\}$ с отличными от нуля членами является бесконечно малой тогда и только тогда, когда последовательность $\{\frac1{\alpha_n}\}$ является бесконечно большой.
\end{problem}

\begin{problem}
    Докажите, что последовательность $\{\alpha_n\}$ положительными членами является бесконечно малой тогда и только тогда,
    когда последовательность $\{\alpha_n^2\}$ является бесконечно малой.
\end{problem}

\begin{problem}
    Последовательности $\{\alpha_n\}$ и $\{\beta_n\}$ бесконечно малые, а последовательность $\{\gamma_n\}$ такова,
    что $\gamma_{2n-1} = \alpha_n$, $\gamma_{2n} = \beta_n$ при любом натуральном $n$.
    Является ли последовательность $\{\gamma_n\}$ бесконечно малой?
\end{problem}

\begin{problem}
    Одна последовательность бесконечно малая, а другая ограниченная.
    Что можно сказать о
    \begin{probparts}
        \item сумме;
        \item произведении;
        \item отношении
    \end{probparts}
    этих последовательностей? (Обязательно ли они бесконечно малые/ограниченные/бесконечно большие?)
\end{problem}

\begin{problem}
    Решите предыдущую задачу в случае, если одна из последовательностей бесконечно малая, а другая бесконечно большая.
\end{problem}

\begin{problem}[$\mathghost$]
    Докажите, пользуясь предыдущими задачами, что следующие последовательности являются бесконечно малыми:
    \begin{probparts}
        \item \mbox{\large$\frac{n^5 + 3}{n^{10}}$};
        \item \mbox{\large$\frac{3n^6 + 2n^4 - n}{n^9 + 7n^5 - 5n^2 - 2}$};
        \item \mbox{\large$\sqrt{\frac{|\sin{3n} + \cos{7n}|}{2n^2 + 3n}}$};
        \item \mbox{\large$\frac{3^n + 4^n}{2^n + 5^n}$}.
    \end{probparts}
\end{problem}
\begin{note}
    Что можно сказать про отношение двух бесконечно больших последовательностей? Избавиться от этой ситуации можно, поделив на что-нибудь одновременно и числитель, и знаменатель.
\end{note}
\end{document}
