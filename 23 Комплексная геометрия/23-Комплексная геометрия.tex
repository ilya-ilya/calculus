\documentclass[a4paper, 12pt, num=23]{listok}
\usepackage{bm}
%\usepackage{scalerel}
%\usepackage{tikz}
%\usepackage{epigraph}
%\usepackage{multicol}
%\usetikzlibrary{matrix}
%\usetikzlibrary{calc}
%\usetikzlibrary{arrows}

\DeclareMathOperator{\id}{id}

%\newcommand*{\hm}[1]{#1\nobreak\discretionary{}{\hbox{$\mathsurround=0pt #1$}}{}} %перед знаком для его дублирования при переносе формулы

\begin{document}
\title{Комплексная геометрия}
\maketitle{}
\begin{problem}
	Найдите геометрическое место точек $z$ на комплексной плоскости, удовлетворяющих условиям: \textbf{а.} $|z - 1| = 1$; \textbf{б.} $|z| = |z + 1|$.
\end{problem}
\begin{problem}
	Докажите, что точки $0$, $z$ и $\frac1z$ лежат на одной прямой.
\end{problem}
\begin{problem}
	Докажите, что уравнение $z + \bar z = z\bar z$ задаёт окружность. Найдите её центр и радиус.
\end{problem}
\begin{problem}
	Нарисуйте множество точек $z \in \Cx$, таких, что $|z - 3| \le 2$ и $|z + 4i| \le 3$.
\end{problem}
\begin{problem}
	Докажите, что произведение диагоналей четырёхугольника не больше суммы произведений его противополжных сторон. Когда это неравенство обращается в равенство?
\end{problem}
\begin{problem}
	Докажите, что для любого числа $a$ преобразование $z \mapsto az$ увеличивает все расстояния в одно и то же число раз. В какое?
\end{problem}
\begin{problem}
	Что можно сказать про это преобразование, если число $a$ действительно? Если $|a| = 1$?
	При каком $a$ это преобразование будет поворотом на $30^\circ$ вокруг начала координат?
\end{problem}
\begin{problem}
	\textbf{а.} Числа $0$ и $z$ являются вершинами правильного треугольника.
	Где может находиться третья его вершина?
	\textbf{б.} На сторонах треугольника с вершинами в точках $u$, $z$, $w$ построены равносторонние треугольники.
	Найдите формулы для их центров (через комплексные числа $u$, $z$, $w$). Докажите, что эти центры образуют равносторонний треугольник.
\end{problem}
\begin{problem}
	При каких $a$ и $b$ преобразование $z \mapsto az + b$ будет поворотом? переносом? осевой симметрией?
\end{problem}
\begin{problem}
	При каких $a$ и $b$ преобразование $z \mapsto a\bar z + b$ является осевой симметрией?
\end{problem}
\begin{problem}
	Докажите, что треугольник с вершинами $0$, $1$, $z$ подобен треугольнику с вершинами $0$, $1$, $\frac1z$.
\end{problem}
\begin{problem}
	Точки $x$, $y$, $z$ комплексной плоскости лежат на одной прямой тогда и только тогда, когда отношение \ldots{} вещественно. Вставьте пропущенную формулу и докажите.
\end{problem}
\begin{problem}
	Найдите геометрическое место точек $z$, для которых число $\frac{z-1}{z-2}$ --- чисто мнимое.
\end{problem}
\begin{problem}
	Докажите, что точки $z$, $w$, $\frac1z$, $\frac1w$ лежат на одной окружности.
\end{problem}
\begin{problem}
	Докажите, что точка $\frac1z$ пробегает окружность, когда $z$ движется по прямой $\Re z = 1$.
\end{problem}
\begin{problem}
	Точки $x$, $y$, $z$, $w$ комплексной плоскости лежат на одной окружности тогда и только тогда, когда отношение \ldots{} вещественно.
	Вставьте пропущенную формулу и докажите.
\end{problem}
\begin{problem}
	Докажите, что уравнение $z\bar z + az + \widebar{az} + c = 0$, где $a$ --- комплексное число,
	а $c$ --- действительное, задаёт пустое множество, прямую или окружность. Как по $a$ и $c$ определить, что именно?
\end{problem}
\begin{problem}[(окружность Аполлония)]
	Найдите геометрическое место точек таких $X$, что $\frac{|AX|}{|BX|} = \mathrm{const}$.
\end{problem}
\begin{problem}[(Степень точки относительно окружности)]
	Докажите, что степень точки $w$ относительно окружности $Az\bar z + Bz - \bar B \bar z + C = 0$ равна
	$w\bar w + \frac B A w - \frac{\bar B}A w + \frac C A$.
\end{problem}
\begin{problem}[(Радикальная ось двух окружностей)]
	Докажите, что геометрическое место точек $w$, степень которых относительно двух неконцентрических окружностей $S_1$ и $S_2$ одинакова, является прямой.
\end{problem}
\begin{problem}[(Радикальный центр трех окружностей)]
	На плоскости даны три окружности $S_1$, $S_2$ и $S_3$.
	Докажите, что если две радикальных оси этих окружностей пересекаются в точке $Q$, то третья радикальная ось также проходит через эту точку.
\end{problem}
\begin{problem}
	Как в комплексных числах записывается образ точки $z$ при инверсии относительно окружности единичного радиуса?
	Относительно произвольной окружности? Докажите основные свойства инверсии через комплексные числа.
\end{problem}
\end{document}
