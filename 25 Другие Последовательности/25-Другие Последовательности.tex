\documentclass[a4paper, 12pt, num=25]{listok}
\usepackage{bm}
%\usepackage{scalerel}
%\usepackage{tikz}
%\usepackage{epigraph}
%\usepackage{multicol}
%\usetikzlibrary{matrix}
%\usetikzlibrary{calc}
%\usetikzlibrary{arrows}

\DeclareMathOperator{\id}{id}

%\newcommand*{\hm}[1]{#1\nobreak\discretionary{}{\hbox{$\mathsurround=0pt #1$}}{}} %перед знаком для его дублирования при переносе формулы

\begin{document}
\title{Другие последовательности}
\maketitle{}
\begin{definition}
	Последовательность называется \textit{неограниченной}, если она не является ограниченной.
\end{definition}
\begin{definition}
	Последовательность $\{a_n\}$ называется \textit{бесконечно большой}, если для любого числа $C > 0$ найдется такое число $k$,
	что при всех натуральных $n$, больших $k$, будет верно неравенство $|a_n| > C$.
\end{definition}
\begin{problem}
	Сформулируйте, не используя отрицания, определения неограниченной и бесконечно большой последовательностей с помощью кванторов.
	Верно ли, что любая бесконечно большая последовательность является неограниченной? А наоборот?
	Рассмотрим множество неограниченных и множество бесконечно больших последовательностей. Пересекаются ли они?
	Является ли одно из них подмножеством другого?
\end{problem}
\begin{problem}
	Сформулируйте, не используя отрицания, определение последовательности, которая
	\textbf{а.} не является неограниченной; \textbf{б.} не является бесконечно большой.
	Запишите их с помощью кванторов.
\end{problem}
\begin{problem}
	Последовательность $\{a_n\}$ бесконечно большая. Верно ли, что она монотонная?
	А последовательность $\{|a_n|\}$?
\end{problem}
\begin{problem}
	Является ли бесконечно большой последовательность, равная \textbf{а.} сумме; \textbf{б.} разности;
	\textbf{в.} произведению; \textbf{г.} отношению бесконечно больших последовательностей?
\end{problem}
\begin{problem}
	Изменятся ли ответы на вопросы предыдущей задачи, если везде заменить бесконечно большие последовательности на неограниченные?
\end{problem}
\begin{definition}
	Пусть $\{n_i\}$  возрастающая последовательность натуральных чисел.
	Последовательность $\{b_i\}$, где $b_i = a_{n_i}$, называется \textit{подпоследовательностью} последовательности $\{a_n\}$.
\end{definition}
\begin{problem}
	\textbf{а.} Докажите, что любая подпоследовательность ограниченной последовательности ограничена.
	Останется ли верным аналогичное утверждение в случае \textbf{б.} монотонной; \textbf{в.} неограниченной; \textbf{г.} бесконечно большой последовательности?
\end{problem}
\begin{problem}
	Докажите, что \textbf{а.} любая последовательность содержит монотонную подпоследовательность;
	\textbf{б.} любая неограниченная последовательность содержит бесконечно большую подпоследовательность.
\end{problem}
\begin{problem}
	\textbf{а.} Докажите, что для любой ограниченной последовательности существует отрезок длины 1,
	в котором находится бесконечно много членов этой последовательности.
	\textbf{б.} Верно ли, что если для некоторой последовательности такого отрезка длины 1 найти нельзя, то эта последовательность  бесконечно большая?
\end{problem}
\begin{problem}
	Докажите, что для любой ограниченной монотонной последовательности найдется отрезок длины 1,
	в котором находятся все члены этой последовательности, начиная с некоторого номера.
\end{problem}
\begin{problem}
	Придумайте две различные последовательности, являющиеся подпоследовательностями друг друга.
\end{problem}
\begin{problem}
	\textbf{а.} Последовательность $\{a_n\}$ такова, что $|a_{n+1} - a_n| \le \frac1{2^n}$ при любом натуральном $n$.
	Может ли эта последовательность быть неограниченной?
	\textbf{б.} Тот же вопрос, если $|a_{n+1} - a_n| \le \frac1n$.
\end{problem}
\begin{problem}
	\textbf{а.} Придумайте такую последовательность натуральных чисел, что любая последовательность натуральных чисел является ее подпоследовательностью.
	\textbf{б.} Можно ли решить аналогичную задачу, если натуральные числа всюду заменить на рациональные числа?
	\textbf{в.} А на действительные?
\end{problem}
\begin{problem}
	Существует ли такая последовательность целых чисел, что любое натуральное число представимо в виде разности двух членов этой последовательности,
	причем единственным образом?
\end{problem}
\begin{problem}
	Докажите, что найдется такое число $a > 0$,
	при котором дробные части всех чисел последовательности $\{a_n\}$ принадлежат отрезку $[1/3; 2/3]$.
\end{problem}
\begin{problem}
	\textbf{а.} Докажите, что у всякой последовательности длины $n^2 + 1$ существует монотонная подпоследовательность длины $n + 1$.
	\textbf{б.} Останется ли верным утверждение задачи в случае последовательности длины $n^2$ при больших значениях $n$?
\end{problem}
\begin{definition}
	Последовательность $\{\alpha_n\}$ называется \textit{бесконечно малой}, если для произвольного положительного числа $\epsilon$
	найдется такое натуральное число $N$, что при любом натуральном $n > N$ будет верно неравенство $|\alpha_n| < \epsilon$.
\end{definition}
\begin{center}
	\textit{В задачах ниже при доказательстве надо явно указывать сколемовскую функцию.}
\end{center}
\begin{problem}
	Запишите в кванторах определение бесконечно малой последовательности и последовательности, не являющейся бесконечно малой.
	Запишите их так, чтобы кванторы существования шли в начале формулы.
\end{problem}
\begin{problem}
	Для последовательности $\{\alpha_n\}$ укажите какой-нибудь номер $N$, начиная с которого для всех членов последовательности верно неравенство
	$|\alpha_n| < \epsilon$, если
	\begin{multienum}{3}
		\item $\alpha_n = \cfrac1n$;
		\item $\alpha_n = \cfrac{{(-1)}^n \cos n}{n^3}$;
		\item $\alpha_n = {0{,}99}^n$;
		\item $\alpha_n = \cfrac{2^n}{n!}$;
		\item $\alpha_n = \cfrac{2n+3}{n^2+2n-1}$;
		\item $\alpha_n = \sqrt{n + 1} - \sqrt n$.
	\end{multienum}
\end{problem}
\begin{problem}
	Верно ли, что \textbf{а.} сумма; \textbf{б.} разность; \textbf{в.} произведение; \textbf{г.} отношение бесконечно малых последовательностей
	также является бесконечно малой последовательностью.
\end{problem}
\begin{problem}
	Пусть последовательности $\{\alpha_n\}$ и $\{\beta_n\}$ являются бесконечно малыми,
	а последовательность $\{\gamma_n\}$ такова, что $\alpha_n \le \gamma_n \le \beta_n$.
	Докажите, что тогда последовательность $\{\gamma_n\}$ также является бесконечно малой.
\end{problem}
\begin{problem}
	Верно ли, что последовательность $\{\alpha_n\}$ с отличными от нуля членами является бесконечно малой тогда и только тогда,
	когда последовательность $\{\frac1{\alpha_n}\}$ является бесконечно большой?
\end{problem}
\begin{problem}
	\textbf{а.} Верно ли, что последовательность $\{\alpha_n\}$ положительными членами является бесконечно малой тогда и только тогда,
	когда последовательность $\{\alpha_n^2\}$ является бесконечно малой?
	\textbf{б.} Верно ли аналогичное утверждение для последовательностей $\{\alpha_n\}$ и $\{\sqrt[3]{\alpha_n}\}$?
\end{problem}
\begin{problem}
	Последовательности $\{\alpha_n\}$ и $\{\beta_n\}$ бесконечно малые, а последовательность $\{\gamma_n\}$ такова,
	что $\gamma_{2n-1} = \alpha_n$, $\gamma_{2n} = \beta_n$ при любом натуральном $n$.
	Является ли последовательность $\{\gamma_n\}$ бесконечно малой?
\end{problem}
\begin{problem}
	Одна последовательность бесконечно малая, а другая ограниченная.
	Что можно сказать о \textbf{а.} сумме; \textbf{б.} произведении; \textbf{в.} отношении этих последовательностей?
\end{problem}
\begin{problem}
	Решите предыдущую задачу в случае, если одна из последовательностей бесконечно малая, а другая бесконечно большая.
\end{problem}
\begin{problem}
	Докажите, пользуясь предыдущими задачами, что следующие последовательности являются бесконечно малыми:
	\begin{multienum}{4}
		\item $\cfrac{n^5 + 3}{n^{10}}$;
		\item $\cfrac{3n^6 + 2n^4 - n}{n^9 + 7n^5 - 5n^2 - 2}$;
		\item $\sqrt{\cfrac{|\sin{3n} + \cos{7n}|}{2n^2 + 3n}}$;
		\item $\cfrac{3^n + 4^n}{2^n + 5^n}$;
	\end{multienum}
\end{problem}
\begin{problem}
	Любую ли последовательность можно представить в виде отношения двух бесконечно малых последовательностей?
\end{problem}
\begin{problem}
	По последовательности $\{\alpha_n\}$ построили последовательность $\{\beta_n\}$ так, что 
	$\beta_n = \alpha_{n + 1} - \frac{\alpha_n}2$ при любом натурального $n$.
	Докажите, что если последовательность $\{\beta_n\}$ оказалась бесконечно малой,
	то и последовательность $\{\alpha_n\}$ также является бесконечно малой.
\end{problem}
\end{document}
