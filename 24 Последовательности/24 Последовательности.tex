\documentclass[a4paper, 12pt, num=24, date=06.11.2019]{listok}
\usepackage{bm}
\usepackage{halloweenmath}
%\usepackage{scalerel}
%\usepackage{tikz}
%\usepackage{epigraph}
%\usepackage{multicol}
%\usetikzlibrary{matrix}
%\usetikzlibrary{calc}
%\usetikzlibrary{arrows}

\DeclareMathOperator{\id}{id}

%\newcommand*{\hm}[1]{#1\nobreak\discretionary{}{\hbox{$\mathsurround=0pt #1$}}{}} %перед знаком для его дублирования при переносе формулы

\begin{document}
\title{Последовательности}
\maketitle{}
\begin{center}
    \textit{Символом $\mathghost$ обозначены задачи, при решении которых могут возникать непреодолимые трудности.
    Если Вы долго думаете над задачей без каких-то продвижений, стоит обратиться к ближайшему преподавателю математического анализа.}
\end{center}
\begin{definition}
    Пусть имеется некоторое непустое множество $U$.
    Любая функция $a \colon \N \to U$ называется \textit{последовательностью элементов множества $U$}.
    Элементы последовательности обычно обозначают так: $a_1 = a(1), a_2 = a(2), \ldots, a_n = a(n), \ldots$,
    а саму последовательность обозначают $\{a_n\}$ или ${\{a_n\}}_{n = 1}^{\infty}$.
\end{definition}
\begin{definition}
    \textit{Числовой последовательностью} называют последовательность элементов какого-нибудь фиксированного множества чисел ($\N, \Z, \Q, \R, \Cx$).
    Обычно, если иное специально не оговорено, говоря о числовых последовательностях, мы будем иметь в виду $\R$-последовательности, реже $\Cx$-последовательности.
\end{definition}
\begin{example}
    Последовательности полных графов разного размера $\{K_n\}$.
\end{example}
\begin{example}
    Последовательность, в которой значение каждого элемента на единицу больше его номера,
    можно записать как $\{n + 1\}$ или $\{a_n\}$, где $a_n = n + 1$.
\end{example}
\begin{definition}
    Последовательность $\{a_n\}$ называется \textit{ограниченной сверху}, если найдётся такое число $C$,
    что при всех натуральных $n$ будет выполнено неравенство $a_n < C$.
\end{definition}
\begin{problem}
    Дайте определение последовательности, ограниченной снизу.
\end{problem}
\begin{problem}
    Приведите пример последовательности
    \begin{probparts}
        \item ограниченной сверху, но не ограниченной снизу;
        \item не ограниченной ни сверху, ни снизу.
    \end{probparts}
\end{problem}
\begin{definition}
    Последовательность $\{a_n\}$ называется \textit{ограниченной}, если она ограничена и сверху, и снизу.
\end{definition}
\begin{problem}
\begin{probparts}
    \item Дайте определение ограниченной последовательности, корректное для комплексных последовательностей и эквивалентное предыдущему для вещественных.
    \item Каков геометрический смысл этих определений?\footnote{Эту задачу можно сдавать устно.}
\end{probparts}
\end{problem}
\begin{problem}
    Приведите пример ограниченной вещественной последовательности, у которой
    \begin{probparts}
        \item есть и наибольший, и наименьший член;
        \item есть наибольший, но нет наименьшего члена;
        \item есть наименьший, но нет наибольшего члена;
        \item нет ни наименьшего, ни наибольшего члена.
    \end{probparts}
\end{problem}
\begin{problem}[$\mathghost$]\label{exfirst}
    Исследуйте на ограниченность следующие последовательности, а также изобразите их на координатной плоскости:
    \begin{multienum}{3}
        \item $a_n = \cfrac{200 - 3n}{101 - 2n}$;
        \item $a_n = \cfrac{10 + 10n - n^2}{n + 1}$;
        \item $a_n = \cfrac{n^2}{2^n}$;
        \item $a_n = \cfrac{100^n}{n!}$;
        \item $a_n = 1,{}01^n$;
        \item $a_n = \frac{1,{}01^n}n$;
        \item $a_n = n \sin{n}$;
        \item $a_n = {(3 - 2i)}^n$;
    \end{multienum}
\end{problem}
\begin{definition}
    Суммой последовательностей $\{a_n\}$ и $\{b_n\}$ называется последовательность $\{c_n\}$ такая,
    что $c_n = a_n + b_n$ при всех $n \in \N$.
    Аналогичным образом определяют разность, произведение, отношение двух последовательностей.
\end{definition}
\begin{problem}
    Известно, что
    \begin{probparts}
        \item сумма;
        \item произведение
    \end{probparts}
    двух последовательностей --- ограниченная последовательность.
    Правда ли, что хотя бы одна из исходных последовательностей ограничена?
\end{problem}
\begin{problem}
    Верно ли, что
    \begin{probparts}
        \item сумма;
        \item разность;
        \item произведение;
        \item отношение
    \end{probparts}
    ограниченных последовательностей --- ограниченная последовательность?
\end{problem}
\begin{problem}[$\mathghost$]
    Являются ли ограниченными последовательности:
    \begin{multienum}{3}
        \item $a_n = \sum\limits_{i = 1}^n \cfrac1{2^i}$;
        \item $a_n = \sum\limits_{i = 1}^n \cfrac1i$;
        \item $a_n = \sum\limits_{i = 1}^n \cfrac1{i(i + 1)}$;
        \item $a_n = \sum\limits_{i = 1}^n \cfrac1{i^2}$;
        \item $a_n = \sum\limits_{i = 1}^n \cfrac1{i!}$;
    \end{multienum}
\end{problem}
\begin{problem}[$\mathghost$]\label{exsecond}
    Являются ли ограниченными следующие последовательности:
    \begin{multienum}{3}
        \item $a_1 = 1, a_{n + 1} = \sqrt{2 + a_n}$;
        \item $a_1 = 1, a_{n + 1} = a_n + \frac1{a_n}$;
        \item $a_n = {(1 + \frac1n)}^n$;
        \item $a_n = \sqrt[n]n$;
    \end{multienum}
\end{problem}
\begin{definition}
    Пусть имеется некоторое множество $U$ (\textit{универсум})
    и некоторое утверждение (\textit{предикат}) $A$ про его элементы.
    То есть для каждого $a \in U$ мы знаем либо, что $A(a)$ верно, либо, что неверно.
    Для построения стандартных математических суждений принято использовать \textit{кванторы всеобщности и существования} следующим образом
    \begin{align*}
        \forall{a \in U} A(a) &\quad\text{читают, как <<Для любого $a \in U$ верно $A(a)$>>},\\
        \exists{a \in U} A(a) &\quad\text{читают, как <<Существует $a \in U$ такой, что верно $A(a)$>>}.
    \end{align*}
    Иногда также выделяют квантор <<существует единственный>> $\existssym$!.
\end{definition}
\begin{example}
    %TODO: Добавить текст к примерам
    \phantom{000}
    \begin{multienum}{2}
        \item $\forall{x \in \N} \exists{y \in \N} y = x + 1$.
        \item $\exists{x \in \R} \forall{y \in \R} x \le |y|$.
        \item $\forall{x \in \N} \exists{y \in \R} x < y$.
        \item $\forall{x \in \N} \exists{y \in \R} y < x$.
    \end{multienum}
\end{example}
\begin{problem}[ (функция Сколема)]
\begin{probparts}
    \item В каких утверждениях из примера можно поменять местами кванторы всеобщности и существования? В каких нельзя? Почему?
    \item Докажите, что любое утверждение вида $\forall{a \in A} \exists{b \in B} C(a, b)$ можно переделать так, чтобы квантор существования стоял в начале.\footnote{%
        Подсказка: внимательно прочитайте первое слово названия задачи.}
    \item Докажите, что любое утверждение можно переписать так, чтобы сначала шли кванторы существования, затем кванторы всеобщности, а затем некоторый предикат.
\end{probparts}
\end{problem}

\begin{problem}
    Запишите с помощью кванторов определения ограниченной снизу, ограниченной, возрастающей, убывающей, невозрастающей, неубывающей последовательностей.
\end{problem}
\begin{definition}
    Последовательность называется \textit{монотонной}, если она является неубывающей либо невозрастающей.
    Последовательность называется \textit{строго монотонной}, если она является либо возрастающей, либо убывающей.
    Очевидно, что строго монотонная последовательность является монотонной.
\end{definition}
\begin{problem}
    Сформулируйте, не используя отрицания, определение последовательности, которая
    \begin{probparts}
        \item не является возрастающей;
        \item не является ограниченной сверху;
        \item не является ограниченной;
        \item не является монотонной.
    \end{probparts}
    Запишите их с помощью кванторов.
\end{problem}
\begin{problem}
    Про каждую из последовательностей задачи~\ref{exfirst} выясните, является ли она монотонной, и найдите, если это возможно, ее наибольший и наименьший члены.
\end{problem}
\begin{problem}
    Про каждую из последовательностей задачи~\ref{exsecond} выясните, является ли она монотонной, и найдите, если это возможно, ее наибольший и наименьший члены.
\end{problem}
\begin{problem}
    Есть ли последовательность, члены которой найдутся в любом интервале числовой оси?
\end{problem}
\begin{problem}
\begin{probparts}
    \item Определим последовательность $\{a_n\}$ следующим образом:
    для любого натурального $n$ пусть $a_n$ равно сумме всех чисел вида $\frac1k$, где $k$ --- натуральное, $1 \le k \le n$.
    Ограничена ли последовательность $\{a_n\}$?
    \item Последовательность $\{b_n\}$ зададим так: для любого натурального $n$ пусть $b_n$ равно сумме всех чисел вида $\frac1k$,
    где $k$ --- натуральное, $1 \le k \le n$ и в десятичной записи числа $k$ нет цифры $9$. Ограничена ли последовательность $\{b_n\}$?
\end{probparts}
\end{problem}
\end{document}
