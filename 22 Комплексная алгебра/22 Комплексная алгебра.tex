\documentclass[a4paper, 12pt, num=22, date=16.09.2019]{listok}
\usepackage{bm}
%\usepackage{scalerel}
%\usepackage{tikz}
%\usepackage{epigraph}
%\usepackage{multicol}
%\usetikzlibrary{matrix}
%\usetikzlibrary{calc}
%\usetikzlibrary{arrows}

\DeclareMathOperator{\id}{id}

%\newcommand*{\hm}[1]{#1\nobreak\discretionary{}{\hbox{$\mathsurround=0pt #1$}}{}} %перед знаком для его дублирования при переносе формулы

\begin{document}
\title{Комплексная алгебра}
\maketitle{}
\begin{abstract}
	Большая часть задач этого листка решается с помощью формулы Муавра.
\end{abstract}
\begin{problem}
	Решите уравнение и отметьте его корни на комплексной плоскости:
	\begin{multienum}{2}
		\item $z^5 = 1$;
		\item $z^4 = 1 - \sqrt3i$.
	\end{multienum}
\end{problem}
\begin{problem}
\begin{probparts}
	\item Сколько решений в зависимости от $n$ имеет уравнение $z^n = 1$ в действительных числах?
	\item Тот же вопрос в комплексных числах.
\end{probparts}
\end{problem}
\begin{definition}
	\textit{Корнем $n$--ой степени} из комплексного числа $w$ называется любое комплексное число $z$, т.ч. $z^n = w$.
\end{definition}
\begin{problem}
	Укажите все корни степени $6$ из $-4$, отметьте их на комплексной плоскости.
\end{problem}
\begin{problem}
	Сколько существует корней степени $n$ из числа $w$? Выразите их через $|w|$ и $\arg w$.
\end{problem}
\begin{problem}
	Докажите, что при любом комплексном $w \ne 0$ все корни степени $n$ из $w$ являются
	вершинами правильного многоугольника на комплексной плоскости.
	Сколько вершин у данного многоугольника,
	где находится его центр и чему равен радиус описанной около него окружности?
\end{problem}
\begin{problem}
	Докажите, что для любого числа\footnote{Еще раз напомним, что речь везде идет про комплексные числа
	(которые могут оказаться и действительными), а многочлены могут иметь комплексные коэффициенты.}
	$z_0$ многочлен $P(z)$ можно и притом единственным образом представить в виде $P(z) = (z - z_0)Q(z) + w$,
	где $Q(z)$ также некоторый многочлен, а $w$ --- некоторое число.
\end{problem}
\begin{definition}
	В условиях предыдущей задачи многочлен $Q(z)$ называется частным,
	а число $w$ --- остатком от деления многочлена $P(z)$ на многочлен $z - z_0$.
\end{definition}
\begin{problem}
	Что называется корнем многочлена?
	Пусть число $z_0$ является корнем многочлена $P(z)$.
	Докажите, что этот многочлен можно представить в виде $P(z) = (z - z_0)Q(z)$,
	где $Q(z)$ --- также некоторый многочлен.
\end{problem}
\begin{problem}
	Докажите, что любой многочлен $n$--ой степени имеет не более $n$ корней.
\end{problem}
\begin{problem}
	Представьте каждый из многочленов в виде произведения многочленов первой и второй степени с действительными коэффициентами:
	\begin{multienum}{3}
		\item\label{usefull} $x^5 - 1$;
		\item $x^4 + x^3 + x^2 + x + 1$;
		\item $x^4 + 8$.
	\end{multienum}
\end{problem}
\begin{problem}
\begin{probparts}
	\item Пользуясь пунктом~\ref{usefull} предыдущей задачи, найдите $\cos{\frac{2\pi}5}$.
	\item Найдите $\cos{\frac \pi 5}$ и $\sin{\frac \pi 5}$,
	а также получите разложение многочлена $x^5 - 1$ в произведение многочленов первой и
	второй степени с действительными коэффициентами, не использующее тригонометрические функции.
	\item Попробуйте найти указанные в данной задаче значения тригонометрических функций
	из геометрических соображений\footnote{нам известен способ их нахождения из равнобедренного треугольника с углом $\frac \pi 5$ при вершине}.
\end{probparts}
\end{problem}
\begin{problem}
\begin{probenum}
	\item Вычислите сумму и произведение всех корней степени $n$ из числа $w$.
	\item Упростите выражение
	\[
		\cos \alpha
		+ \cos{(\alpha + \frac{2\pi}7)} + \cos{(\alpha + \frac{4\pi}7)} + \cos{(\alpha + \frac{6\pi}7)}
		+ \cos{(\alpha - \frac{2\pi}7)} + \cos{(\alpha - \frac{4\pi}7)} + \cos{(\alpha - \frac{6\pi}7)}.
	\]
\end{probenum}
\end{problem}
\begin{problem}
	Найдите общую формулу $n$--ого члена последовательности, заданной рекуррентно, если
	\begin{multienum}{2}
		\item $a_1 = a_2 = 1, a_n = 5a_{n-1} - 6a_{n-2}$;
		\item $b_1 = b_2 = 1, b_n = 2b_{n-1} - 3b_{n-2}$.
	\end{multienum}
\end{problem}
\begin{problem}
	Найдите остаток от деления многочлена $x^{100} + 3x + 2$ на
	\begin{multienum}{2}
		\item $x^2 - 3x + 2$;
		\item $x^2 - 2x + 2$.
	\end{multienum}
\end{problem}
\begin{problem}
	Докажите, что многочлен $x^{3k} + x^{3l+1} + x^{3m+2}$ делится на $1 + x + x^2$ при любых натуральных $k, l, m$.
\end{problem}
\begin{problem}
	При каких значениях $m$ многочлен ${(x + 1)}^m + x^m + 1$ делится на $1 + x + x^2$?
\end{problem}
\begin{problem}
	Найдите суммы $C_n^0 - C_n^2 + C_n^4 - \ldots$ и $C_n^1 - C_n^3 + C_n^5 - \ldots$
\end{problem}
\begin{problem}
	Найдите суммы
	\begin{multienum}{2}
		\item $\sin \alpha + 2 \sin{2\alpha} + \ldots + 2^{n-1} \sin{n\alpha}$;
		\item $\cos \alpha + \cos{(x + \alpha)} + \ldots + \cos{(nx + \alpha)}$.
	\end{multienum}
\end{problem}
\begin{problem}
	Докажите, что многочлен $x^{44} + x^{33} + x^{22} + x^{11} + 1$ делится на $x^4 + x^3 + x^2 + x + 1$.
\end{problem}
\begin{problem}
	Докажите, что если $f_1(x^3) + x f_2(x^3)$ делится на $1 + x + x^2$, то $f_1(x)$ и $f_2(x)$ делятся на $x - 1$.
\end{problem}
\begin{problem}
	Найдите суммы
	\begin{multienum}{2}
		\item $C_n^0 + C_n^4 + C_n^8 + C_n^{12} + \ldots$;
		\item $C_n^i + C_n^{i + 3} + C_n^{i + 6} + \ldots$, $i = 0, 1, 2$.
	\end{multienum}
\end{problem}
\begin{problem}
	Найдите суммы
	\begin{multienum}{2}
		\item $1 + 2 \cos \alpha + 3 \cos{2\alpha} + \ldots + n \cos{(n−1)\alpha}$;
		\item $\cos^2 \alpha + \cos^2 {2\alpha} + \ldots + \cos^2 {n\alpha}$.
	\end{multienum}
\end{problem}
\begin{problem}
\begin{probparts}
	\item Разложите на множители первой и второй степени с действительными коэффициентами многочлен $x^{2n} - 1$.
	Пользуясь полученным разложением, найдите $\sin{\frac{\pi}{2n}}\sin{\frac{2\pi}{2n}}\ldots\sin{\frac{(n - 1)\pi}{2n}}$ и
	$\cos{\frac{\pi}{2n}}\cos{\frac{2\pi}{2n}}\ldots\cos{\frac{(n - 1)\pi}{2n}}$.
	\item Найдите $\sin{\frac{\pi}{2n + 1}}\sin{\frac{2\pi}{2n + 1}}\ldots\sin{\frac{n\pi}{2n + 1}}$.
\end{probparts}
\end{problem}
\begin{problem}
\begin{probparts}
	\item Докажите, что число $z = \frac35 + \frac45 i$ не является корнем $n$--ой степени из $1$ ни при каком $n$.
	\item Докажите, что градусная мера любого острого угла треугольника со сторонами $3$, $4$, $5$ есть число иррациональное.
	\item Докажите, что острый угол любого прямоугольного треугольника с целочисленными сторонами выражается иррациональным числом градусов.
\end{probparts}
\end{problem}
\begin{problem}
	Найдите сумму и произведение $s$--тых степеней всех корней уравнения $z^n -1 = 0$ в зависимости от $s$ и $n$.
\end{problem}
\end{document}
