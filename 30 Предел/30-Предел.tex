\documentclass[a4paper, 12pt, num=30]{listok}
\usepackage{bm}
\usepackage{marvosym}
%\usepackage{scalerel}
%\usepackage{tikz}
\usepackage{epigraph}
%\usepackage{multicol}
%\usetikzlibrary{matrix}
%\usetikzlibrary{calc}
%\usetikzlibrary{arrows}

\DeclareMathOperator{\id}{id}

%\newcommand*{\hm}[1]{#1\nobreak\discretionary{}{\hbox{$\mathsurround=0pt #1$}}{}} %перед знаком для его дублирования при переносе формулы

\begin{document}
\title{Предел}
\maketitle{}

\epigraph{%
    Бесконечность --- не предел!
}{%
    Buzz Lightyear, space ranger
}
\begin{problem}
    Приведите пример последовательности, которая имеет
    \begin{probparts}
        \item ноль;
        \item одну;
        \item две;
        \item $N$, для некоторого фиксированного $N \in \N$;
        \item счётное количество
    \end{probparts}
    предельных точек.
\end{problem}
\begin{definition}
    Последовательность $\{a_n\}$ называется \textit{сходящейся}, если существует такая точка $A$, что в любой её окрестности содержатся все кроме конечного числа члены $a_n$.
    Обозначение: $A = \lim\limits_{n \to +\infty} a_n$ или $a_n \to A$ при $n \to +\infty$.

    Последовательности, которые не являющиеся сходящимися, называют \textit{расходящимися}.
\end{definition}
\begin{problem}
    Докажите, что $A = \lim\limits_{n \to +\infty} a_n$ тогда и только тогда, когда существует такая бесконечно малая $\{\alpha_n\}$, что $a_n = A + \alpha_n$.
\end{problem}
\begin{problem}
    Докажите, что $A$ --- предел последовательности $a_n$ тогда и только тогда, когда
    \[
        \forall{\epsilon > 0}\exists{N \in \N}\forall{n > N} |a_n - A| < \epsilon.
    \]
\end{problem}
\begin{problem}
    Докажите, что $A$ --- предельная точка последовательности $\{a_n\}$ тогда и только тогда, когда существует подпоследовательность $a_{n_k} \to A$ при $n_k \to +\infty$.
\end{problem}
\begin{definition}
    Последовательность $\{a_n\}$ называется \textit{фундаментальной}, если для любого $\epsilon > 0$ существует $N \in \N$ такое,
    что для любых $n, m > N$ имеет место  $|a_n - a_m| < \epsilon$.
\end{definition}
\begin{problem}[(Критерий Коши)]
    Докажите, что $a \colon \N \to \R$ сходится если и только если является фундаментальной.
\end{problem}
\begin{problem}
    Приведите пример фундаментальной расходящейся $a \colon \N \to \Q$.
\end{problem}
\begin{problem}[(Теорема Больцано--Вейерштрасса)]
    Докажите, что в $\R$ сходится любая ограниченная монотонная последовательность.
\end{problem}
\begin{problem}
    \begin{probparts}
        \item Докажите, что всякая сходящаяся ограничена.
        \item Приведите пример ограниченной расходящейся последовательности.
    \end{probparts}
\end{problem}
\begin{problem}
    Пусть $A$ --- предел последовательности $\{a_n\}$.
    \begin{probparts}
        \item Верно ли, что если $A > 0$, то все члены $\{a_n\}$, начиная с некоторого, положительны?
        \item Докажите, что если в $\{a_n\}$ бесконечно много положительных и отрицательных членов, то $A = 0$.
    \end{probparts}
\end{problem}
\begin{problem}
    Последовательности $\{a_n\}$ и $\{b_n\}$ имеют пределы $A$ и $B$ соответственно.
    Докажите, что тогда
    \begin{probenum}
        \item $\lim\limits_{n \to +\infty} (a_n \pm b_n) = A \pm B$;
        \item $\lim\limits_{n \to +\infty} (a_n  b_n) = A B$;
        \item если $B ̸\ne 0$ и все элементы последовательности $\{b_n\}$ отличны от нуля, то
            $\lim\limits_{n \to +\infty} \frac{a_n}{b_n} = \frac AB$.
    \end{probenum}
\end{problem}
\begin{definition}
    Говорят, что почти все члены последовательности удовлетворяют некоторому условию,
    если существует лишь конечное число членов последовательности, не удовлетворяющих этому условию.
\end{definition}
\begin{problem}
    Пусть последовательности $\{a_n\}$ и $\{b_n\}$ сходятся.
    Докажите, что если почти для всех $n \in \N$ выполняется условие
    \begin{probparts}
        \item $a_n = b_n$, то $\lim\limits_{n \to +\infty} a_n = \lim\limits_{n \to +\infty} b_n$;
        \item $a_n \ge b_n$, то $\lim\limits_{n \to +\infty} a_n \ge \lim\limits_{n \to +\infty} b_n$.
        \item Останется ли верным последнее утверждение, если в нем все знаки нестрогого неравенства заменить на знаки строгого неравенства?
    \end{probparts}
\end{problem}
\begin{problem}[(Лемма о двух милиционерах)]
    Пусть последовательности $\{a_n\}$, $\{b_n\}$, $\{c_n\}$ таковы, что почти для всех $n \in \N$ выполнено неравенство
    $a_n \le c_n \le b_n$ и $\lim\limits_{n\to+\infty} a_n = \lim\limits_{n\to+\infty} b_n = A$.
    Докажите, что тогда $\lim\limits_{n\to+\infty} c_n = A$.
\end{problem}
\begin{problem}
\begin{probparts}
    \item Алиса записала определение последовательности, имеющей предел, следующим образом:
        <<$
            \exists{N}\forall{\epsilon > 0}\forall{n > N} |a_n - A| < \epsilon.
        $>>
        Опишите множество последовательностей, которые задает данное определение.
    \item Выполните это же задание для определения Боба:
        <<$
            \exists{\epsilon > 0} \forall N \forall{n > N} |a_n - A| < \epsilon.
        $>>
\end{probparts}
\end{problem}
\end{document}
